\section{Introduction}
\begin{frame} {Goals and Significance of the Research}
  Goals:
  \begin{itemize}
    \item Investigate the stability of transonic plasma flow in magnetic nozzle.
    \item Understand the plasma behavior near the nozzle throat.
    \item Dealing with singularity at nozzle throat both analytically and numerically.
  \end{itemize}

  Significance:
  \begin{itemize}
    \item Understand better the transonic plasma flow in magnetic nozzle.
    \item Addresses the complexity introduced by the singularity in the governing equations.
    \item Offer insights into plasma flow in magnetic mirror configuration.
  \end{itemize}
\end{frame}

\begin{frame}{Linear Instability of Plasma Flow}
  \begin{itemize}
    \item The instability of plasma flow refers to the tendency of a plasma system to deviate from a stable, equilibrium state and exhibit perturbations or fluctuations in its behavior. \cite{chen_introduction_2016}
    \item To investigate linear instability, we assume oscillating perturbed quantities, $\tilde{n}, \tilde{v} \sim \exp(-i\omega t)$.
          \begin{enumerate}
            \item If $\Im(\omega) > 0$, then it is unstable flow since the perturbations grow exponential in time, $\exp(\Im(\omega) t)$.
            \item If $\Im(\omega) <=0$, then it is stable flow since the perturbations decay/unchanged in time.
          \end{enumerate}
    \item In this research we will focus on the linear instability only.
  \end{itemize}
\end{frame}

\begin{frame}{Magnetic Nozzle}
  \begin{itemize}
    \item A magnetic nozzle is a device that uses a magnetic field to shape and control the flow of charged particles in a plasma propulsion system.
    \item Instabilities may affect magnetic nozzle operation and the resulting thrust. \cite{kaganovich_2020_physics}
  \end{itemize}
  \begin{figure}[htbp]
    \centering
    \begin{subfigure}{0.45\textwidth}
      \centering
      \includegraphics[width=0.8\linewidth]{figures/magnetic-nozzle}
      \caption{Simplified representation of magnetic nozzle.}
    \end{subfigure}%
    \begin{subfigure}{0.45\textwidth}
      \centering
      \includegraphics[width=0.8\linewidth]{figures/magnetic-field}
      \caption{A simplified magnetic field of magnetic nozzle.}
    \end{subfigure}
    \caption{Simplified representation of magnetic nozzle. Length is normalized.}
    \label{fig:magnetic-nozzle}
  \end{figure}
\end{frame}

\begin{frame}{Governing Equations}
  The nondimensionalized governing equations for the plasma flow in magnetic nozzle are \cite{smolyakov_quasineutral_2021}
  \begin{align}
     & \text{Cons. of Den.} &  & \pdv{n}{t} + n\pdv{v}{z} + v\pdv{n}{z} - nv\frac{\partial_z B}{B} = 0 \\
     & \text{Cons. of Mom.} &  & n\pdv{v}{t} + nv\pdv{v}{z} = -\pdv{n}{z}
  \end{align}
  where $n,v$ are density and velocity, respectively.

  The equilibrium quantities $n_0, v_0$ must satify the condition,
  \begin{align}
     & \pdv{z}(\frac{n_0v_0}{B}) = 0 \label{eq:equilibrium-convervation-of-mass}                    \\
     & v_0\pdv{v_0}{z} = -\frac{1}{n_0}\pdv{n_0}{z} \label{eq:equilibrium-convervation-of-momentum}
  \end{align}
\end{frame}

\begin{frame}{Equilibrium Velocity Profiles}
  \begin{figure}
    \begin{center}
      \includegraphics[width=0.6\textwidth]{figures/velocity-profiles.png}
    \end{center}
    \caption{Velocity if normalized to sound speed. There are 4 different cases for velocity profile: subsonic, supersonic, accelerating and decelerating case. In this research we are focusing on the accelerating velocity profile.}
    \label{fig:velocity-profiles}
  \end{figure}
\end{frame}

\begin{frame}{Polynomial Eigenvalue Problem}
  By linearizing the governing equations, and assume oscillating perturbed quantities, $\tilde{n}, \tilde{v} \sim \exp(-i\omega t)$. We can derive the following polynomial eigenvalue problem,

  \begin{equation}
    \begin{aligned}
        & \omega^2 \tilde{v}                                                 \\
      + & 2i\omega\left(v_0\pdv{}{z} + \pdv{v_0}{z}\right) \tilde{v}         \\
      + & \left[ \textcolor{red}{(1-v_0^2)\pdv[2]{}{z}}
      -\left(3v_0 + \frac{1}{v_0}\right)\pdv{v_0}{z}\pdv{}{z} \right.        \\
        & - \left. \left(1-\frac{1}{v_0^2}\right)\left(\pdv{v_0}{z}\right)^2
        - \left(v_0+\frac{1}{v_0}\right)\pdv[2]{v_0}{z} \right]\tilde{v}
      = 0
    \end{aligned}
    \label{eq:polynomial_eigenvalue_problem}
  \end{equation}

  Notice that the highest derivative term (labeled in red) vanishes at the throat of the nozzle, $z=0$. Spectral method fails to resolve eigenmodes because of the existence of this singularity.
\end{frame}
