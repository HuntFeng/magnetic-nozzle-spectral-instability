\documentclass{report}
\usepackage{amsmath, amssymb, amsthm}
\usepackage{physics}
\usepackage{float, subcaption, graphicx}
\usepackage{hyperref} % comment this in school template

\theoremstyle{plain}
\newtheorem{proposition}{Proposition}

\theoremstyle{definition}
\newtheorem{definition}{Definition}

\newtheorem{theorem}{Theorem}

\newtheorem*{remark}{Remark}


\title{Instability In Magnetic Nozzle and Spectral Pollution}
\author{Hunt Feng}

%%%%%%%%%%%%%%%%%%%%%%%%%%%%%%%%%%%%%%%%%%%%%%%%%%%%%%%%%%%%%%%%
% END OF FRONTMATTER SECTION
%%%%%%%%%%%%%%%%%%%%%%%%%%%%%%%%%%%%%%%%%%%%%%%%%%%%%%%%%%%%%%%%
\begin{document}
% Typeset the title page
\maketitle

\tableofcontents % remove this when using school template
% move abstract before the document in school template
\abstract{
    Spectral theory is a common technique for analyzing the instability of a dynamical system. By discretizing the linearized equations motion of magnetic nozzle, the instability problem becomes an algebraic eigenvalue problem. Given Dirichlet boundary condition, we found that the flow in magnetic nozzle is stable. Different discretizations, such as finite difference, finite element and spectral element method agree with each other. By studying the convergence of different modes, we successfully eliminated the spurious unstable modes occur in supersonic and transonic cases. 
}

%%%%%%%%%%%%%%%%%%%%%%%%%%%%%%%%%%%%%%%%%%%%%%%%%%%%%%%%%%%%%%%%
% FIRST CHAPTER OF THESIS BEGINS HERE
%%%%%%%%%%%%%%%%%%%%%%%%%%%%%%%%%%%%%%%%%%%%%%%%%%%%%%%%%%%%%%%%
\chapter{Introduction}
\section{Flow in Magnetic Nozzle}
\subsection{Magnetic Nozzle}
A magnetic nozzle is a device that uses a magnetic field to shape and control the flow of charged particles in a plasma propulsion system, see Fig.\ref{fig:magnetic-nozzle}. By employing the magnetic mirror configuration, the magnetic nozzle can effciently direct and accelerate the plasma flow, generating thrust for propulsion. The magnetic field in the nozzle helps collimate and focus the plasma exhast, increasing its velocity and enhancing the performance of the propulsion system.

\begin{figure}[htbp]
  \centering
  \includegraphics[width=0.7\linewidth]{../../thesis/img/introduction/magnetic-nozzle}
  \caption{A simplified model of the magnetic nozzle}
  \label{fig:magnetic-nozzle}
\end{figure}

\subsection{Magnetic Field in Magnetic Nozzle}
In 1D problem, the magnetic field can be modeled as
\[ B(z) = B_0 \left[1 + R\exp(-\left(\frac{x}{\delta}\right)^2)\right] \]
where $1+R$ is the magnetic mirror ratio, and $\delta$ determines the spread of the magnetic field. It is shown in Fig.(\ref{fig:magnetic-field}).
\begin{figure}[H]
	\centering
	\includegraphics[width=0.7\linewidth]{../../thesis/img/introduction/magnetic-field}
	\caption{This is the magnetic field in nozzle with mirror ratio $1+R=B_{max}/B_{min}=2.5$, and the spread of magnetic field, $\delta=0.1/0.3=0.\bar{3}$. }
	\label{fig:magnetic-field}
\end{figure}

\subsection{Velocity Profile at Equilibrium}
Let $n_0$ and $v_0$ be the density and velocity at equilibrium (stationary solution), we know that $\pdv*{n_0}{t}=0$ and $\pdv*{v_0}{t}=0$, therefore $n_0$ and $v_0$ satisfy the so-called equilibrium condition,
\begin{align*}
	&\pdv{z}(\frac{n_0v_0}{B}) = 0 \\
	&v_0\pdv{v_0}{z} = -c_s^2\frac{1}{n_0}\pdv{n_0}{z} 
\end{align*}

Let $M(z) = v_0(z)/c_s$ be the mach number (nondimensionalized velocity). The equations of motion become
\begin{align*}
	&B\pdv{z}(\frac{n_0M}{B}) = 0\\
	&M\pdv{M}{z} = -\frac{1}{n_0}\pdv{n_0}{z}
\end{align*}
Substitute $\frac{1}{n_0}\pdv*{n_0}{z}$ using first equation, the conservation of momentum becomes
\[ (M^2-1)\pdv{M}{z} = -\frac{M}{B}\pdv{B}{z} \]

Notice that there is a singularity at $M=1$, the sonic speed.

This is a separable equation, integrate it and use the conditions at midpoint $B(0)=B_m, M(0)=M_m$ we get
\[ M^2e^{-M^2} = \frac{B^2}{B_m^2}M_m^2e^{-M_m^2} \]
We can now express $M$ using the Lambert W function,
\[ M(z) = \left[ -W_k\left(-\frac{B(z)^2}{B_m^2}M_m^2e^{-M_m^2}\right) \right]^{1/2} \]
where the subscript $k$ of $W$ stands for branch of Lambert W function. When $k=0$, it is the subsonic branch; When $k=-1$, it is the supersonic branch. Below shows a few cases of the solution.
\begin{itemize}
	\item $M_m < 1, k=0$, subsonic velocity profile.
	\item $M_m = 1$, $k=0$ for $x<0$ and $k=-1$ for $x>0$, accelerating profile
	\item $M_m = 1$, $k=-1$ for $x<0$ and $k=0$ for $x>0$, decelerating profile
	\item $M_m > 1, k=-1$, supersonic velocity profile
\end{itemize}
 Fig.(\ref{fig:velocity_profiles}) shows some cases of the solution.
\begin{figure}[H]
	\centering
	\includegraphics[width=0.7\linewidth]{../../thesis/img/introduction/velocity-profiles}
	\caption{The velocity profile in the magnetic nozzle is completely determined by $M_m$, the velocity at the midpoint, $z=0$. For the transonic velocity profiles, $M_m$ alone is not enough to determine the profile, we need to specify the branch of Lambert W function to determine whether it is accelerating or decelerating.}
	\label{fig:velocity_profiles}
\end{figure}

\subsection{Flow in Similar Configuration: Bondi-Parker Flow}
Bondi derived a steady-state solution for accretion flow which is governed by Bernoulli's equation in sperical symmetry around a point mass in 1952. Then Parker solved a similar problem but with outward wind in 1958. \cite{aikawa_stability_1979,bondi_spherically_1952,keto_stability_2020} The equilibrium velocity profiles in such configuration are shown in Fig.\ref{fig:BP-flow-velocity}.

\begin{figure}[htbp]
    \centering
    \includegraphics[width=0.7\textwidth]{../../thesis/img/introduction/steady-state-BP-flow}
    \caption{Representative trajectories of the steady-state BP flow in non-dimensional units. \cite{keto_stability_2020} The upward pink line represents a outward wind, it accelerates from subsonic to supersonic. The downward pink line represents an accretion flow, it accelerates towards the mass point. The green lines below the pink lines represent subsonic flows, and the green lines above represent supersonic flows. Orange lines are physically impossible scenarios.}
    \label{fig:BP-flow-velocity}
\end{figure}

Solar wind is an example of Bondi-Parker flow. The solar wind is a stream of charged particles, primarily electrons and protons, flowing outward from the Sun. 

\section{Instability of Plasma Flow}
In this section, plasma instability will be introduced and from that we will discuss the importance of this research.

The instability of plasma flow refers to the tendency of a plasma system to deviate from a stable, equilibrium state and exhibit perturbations or fluctuations in its behavior. These instabilities can arise from various factors, such as the interaction of particles with electromagnetic fields, collective effects, or the presence of gradients in plasma parameters.

\begin{figure}[htbp]
	\centering
	\includegraphics[width=0.7\linewidth]{../../thesis/img/introduction/stability-visualization}
	\caption{Mechanical analogy of various types of equilibrium. \cite{chen_introduction_2016}}
	\label{fig:stability-visualization}
\end{figure}

Plasma flow in magnetic mirror configurations have been studied extensively in plasma physics due to its frequent precense in many areas such as the accretion flow \cite{jockers_stability_1968,aikawa_stability_1979}, and magnetic nozzle\cite{smolyakov_quasineutral_2021}. However, the stability of these configurations remains a debatable subject.

\section{Goal of this Thesis}
The goal of this thesis is to study the instabilities of the magnetic mirror configuration given certain boundary conditions and equilibrium velocity profiles.

To achieve the goal, first we need to study the spectral method for solving the instability problem. To use spectral method, it is necessary to understand different discretizations of the operators, such as finite difference, finite element and spectral element method.

Once the spectral method is introduced, we can use it to study the instability of plasma flow in magnetic nozzle as an eigenvalue problem. We can obtain results by using different discretization techniques. By comparing the results from different approach, we can increase the credibility of the true solution.

However, spectral method is not suitable when the equilibrium velocity profile is transonic due to the precense of singularity at the sonic point. We need to solve the singular perturbation problem around the singularity analytically. Then starting from the singular point, we can use shooting method to find eigenvalues.

\section{Outline of the thesis}
The thesis will be divided into several chapters. In chapter 2, spectral method will be introduced. Chapter 3 will focus on the physics of flow in magnetic mirror configuration and derive the governing equations for charged particles, the linearized equations of motions. Following this, we will analyze the problem analytically in chapter 4. Moving to chapter 5, numerical experiments will be conducted. The conclusion will be made in chapter 6.

\chapter{Governing Equations}
\section{Governing Equations for Flow in Magnetic Nozzle}
In this section, we will derive the governing equations of the flow in magnetic nozzle, starting from the fluid description for plasma.

In magnetic nozzle, the magnetic field is along the nozzle, which we denote as z-axis. Due to Lorentz force, the charged particles gyrates about the magnetic field lines. Because the magnetic moment is invariant in such situation (\textbf{reference}). The fluid velocity of particles can be written as $\mathbf{v} = v\mathbf{B}/B$, meaning that the particles move along the magnetic field lines. Therefore the conservation of density 
\[ 
\pdv{n}{t} + \div(n\mathbf{v}) = 0 
\Rightarrow 
\pdv{n}{t} + B\pdv{z}(\frac{nv}{B}) = 0  
\]
In the derivation, $\div{\mathbf{B}} = 0$ is used.

To derive the second governing equation, we start from the conservation of momentum, 
\[ \pdv{v}{t} + v\pdv{v}{z} = -\frac{1}{\rho}\grad{p} \]
Let $\grad{p} = k_BT\pdv*{n}{z}$, we have
\[ \pdv{v}{t} + v\pdv{v}{z} = -c_s^2\frac{1}{n}\pdv{n}{z} \]
where $c_s^2 = k_BT/m$ is the square of sound speed.

Therefore the dynamics of the flow in magnetic nozzle can be characterized by the conservation of density and momentum,
\begin{align*}
	&\pdv{n}{t} + B\pdv{z}(\frac{nv}{B}) = 0\\
	&\pdv{v}{t} + v\pdv{v}{z} = -c_s^2\frac{1}{n}\pdv{n}{z}
\end{align*}

\section{Linearized Equations}
For convenience, we nondimensionalize the governing equations by normalizing the velocity to $c_s$, $v\mapsto v/c_s$, $z$ to system length $L$, $z \mapsto z/L$ and time $t\mapsto c_s t/L$. The governing equations become
\begin{align}
    &\pdv{n}{t} + n\pdv{v}{z} + v\pdv{n}{z} - nv\frac{\partial_z B}{B} = 0 \\
    &n\pdv{v}{t} + nv\pdv{v}{z} = -\pdv{n}{z}
\end{align}
and the nondimensionalized equilibrium condition is
\begin{align}
    &\pdv{z}(\frac{n_0v_0}{B}) = 0 \label{eq:equilibrium-convervation-of-mass}\\
    &v_0\pdv{v_0}{z} = -\frac{1}{n_0}\pdv{n_0}{z} \label{eq:equilibrium-convervation-of-momentum}
\end{align}

Now we are going to derive an important intermediate result, the linearized governing equations.
    Let $n = n_0(z) + \tilde{n}(z,t)$ and $v = v_0(z) + \tilde{v}(z,t)$, where $\tilde{n}$ and $\tilde{v}$ are small perturbed quantities. The linearized governing equations are
  \begin{align}
      &\frac{1}{n_0}\pdv{\tilde{n}}{t} 
      + \pdv{\tilde{v}}{z} + v_0\tilde{Y} + \tilde{v}\frac{\partial_z n_0}{n_0} - \tilde{v}\frac{\partial_z B}{B} = 0 
      \label{eq:linearized-conservation-of-mass}
      \\
      &\pdv{\tilde{v}}{t} + \pdv{(v_0\tilde{v})}{z} = -\tilde{Y}
      \label{eq:linearized-conservation-of-momentum}
  \end{align}
  where 
  \[ \tilde{Y} \equiv \frac{1}{n_0}\pdv{\tilde{n}}{z} - \frac{\partial_z n_0}{n_0^2}\tilde{n} = \pdv{z}(\frac{\tilde{n}}{n_0}) \]

\section{Polynomial Eigenvalue Problem}
In order to investigate the instability of magnetic nozzle, we need formulate it as an eigenvalue problem. To do that, we assume the perturbed density and velocity are oscillatory, i.e. $\tilde{n}, \tilde{v} \sim \exp(-i\omega t)$, where $\omega$ is the oscillation frequency of the perturbed quantities. This frequency can be a complex number. If $\omega = \omega_r +i \omega_i$, then the perturbed quantities becomes $\tilde{n} \sim \exp(\omega_i t)\exp(i\omega_r t)$, which means it grows exponentially with time.

Let $\tilde{n}\sim \exp(-i\omega t)$ and $\tilde{v} \sim \exp(-i\omega t)$, then we have the polynomial eigenvalue problem
\begin{equation}
  \begin{aligned}
    &\omega^2 \tilde{v} \\
    +& 2i\omega\left(v_0\pdv{}{z} + \pdv{v_0}{z}\right) \tilde{v} \\
    +& \left[ (1-v_0^2)\pdv[2]{}{z} 
      -\left(3v_0 + \frac{1}{v_0}\right)\pdv{v_0}{z}\pdv{}{z} 
      -\left(1-\frac{1}{v_0^2}\right)\left(\pdv{v_0}{z}\right)^2 
    - \left(v_0+\frac{1}{v_0}\right)\pdv[2]{v_0}{z} \right]\tilde{v}
    = 0
  \end{aligned}
  \label{eq:polynomial-eigenvalue-problem}
\end{equation}



\chapter{Methodology} \label{chap:methodology}

\section{Spectral Method}
\subsection{Solve as Matrix Eigenvalue Problem}
\subsection{Discretization of Operators}
\subsection{Spectral Pollution}


\section{Shooting Method}
\subsection{Solve as Root Finding Problem}
\subsection{Expansion at Singularity}
\chapter{Numerical Experiments}
\section{Constant Velocity Case - Subsonic}
\begin{figure}[H]
  \centering
  \begin{subfigure}{0.45\textwidth}
    \includegraphics[width=0.9\linewidth]{../../thesis/img/numerical-experiments/fixed-fixed/constant-v-v0=0.5}
    \caption{Dirichlet boundary, all modes are stable.}
  \end{subfigure}%
  \begin{subfigure}{0.45\textwidth}
    \includegraphics[width=\linewidth]{../../thesis/img/numerical-experiments/fixed-open/constant-v-v0=0.5}
    \caption{Fixed-open boundary, all modes are stable.}
  \end{subfigure}
  \caption{Showing the first 5 eigenvalues. In the Dirichlet boundary case, all methods are close to the exact eigenvalues. Meanwhile, finite-difference method has higher accuracy than finite-element method in fixed-open case.}
  \label{fig:constant-v-subsonic}
\end{figure}

\section{Constant Velocity Case - Supersonic}
\begin{figure}[H]
  \begin{subfigure}{0.45\textwidth}
    \centering
    \includegraphics[width=0.9\linewidth]{../../thesis/img/numerical-experiments/fixed-fixed/constant-v-v0=1.5}
    \caption{Dirichlet boundary, filtered modes are stable.}
  \end{subfigure}%
  \begin{subfigure}{0.45\textwidth}
    \includegraphics[width=\linewidth]{../../thesis/img/numerical-experiments/fixed-open/constant-v-v0=1.5}
    \caption{Fixed-open boundary, all modes are unstable.}
  \end{subfigure}
  \caption{Showing the first 5 eigenvalues. In the Dirichlet boundary case, all methods are close to the exact eigenvalues. Meanwhile, finite-difference method has higher accuracy than finite-element method in fixed-open case.}
  \label{fig:constant-v-supersonic}
\end{figure}

\section{Error}
Because the existence of exact solution to problems Eq.(\ref{eq:polynomial-eigenvalue-problem}). The case with constant velocity profile is used as a sanity check. It allows us to verify the correctness of each method's implementation. This also serves as a reference to the accuracy spectral methods can achieve.

From Fig.\ref{fig:constant-v-subsonic} and Fig.\ref{fig:constant-v-supersonic}, we see that the order of growth rates is about $~10^{-14}$ for both subsonic and supersonic cases if the boundary condition is Dirichlet. We will use it a reference to the accuracy of our numerical methods. If a method produces growth rates with order close to $10^{-14}$, we consider the growth rates to be 0.

\begin{table} [H]
	\centering
	\caption{Relative error of each eigenvalue.}
	\begin{tabular}{|c|c|c|c|c|c|}
		\hline
		$v_0=0.5$   & 1 & 2 & 3 & 4 & 5 \\
		\hline
		FD & 2.827e-05 & 1.130e-04 & 2.541e-04 & 4.512e-04 & 7.040e-04 \\
		\hline
		FE & 0.005 & 0.005 & 0.006 & 0.008 & 0.010  \\
		\hline
		SE & 2.896e-05 & 1.157e-04 & 2.603e-04 & 4.626e-04 & 7.217e-04 \\
		\hline
	\end{tabular}
	\begin{tabular}{|c|c|c|c|c|c|}
		\hline
		$v_0=1.5$   & 1 & 2 & 3 & 4 & 5 \\
		\hline
		FD & 0.001 & 0.005 & 0.010 & 0.019 & 0.030 \\
		\hline
		FE & 0.006 & 0.010 & 0.019 & 0.029 & 0.043  \\
		\hline
		SE & 0.001 & 0.005 & 0.011 & 0.019 & 0.030 \\
		\hline
	\end{tabular}
	\label{table:eigenvalue-error-fixed-fixed}
\end{table}

\begin{table} [H]
	\centering
	\caption{Relative error of each eigenvalue. Notice that the ground mode for subsonic case is non-zero.}
	\begin{tabular}{|c|c|c|c|c|c|}
		\hline
		$v_0=0.5$   & 0 & 1 & 2 & 3 & 4 \\
		\hline
		FD & 1.209e-05 & 3.458e-05 & 5.775e-05 & 8.153e-05 & 1.061e-04 \\
		\hline
		FE & 8.090e-05 & 2.007e-04 & 2.981e-04 & 6.596e-04 & 1.821e-03  \\
		\hline
	\end{tabular}
	\begin{tabular}{|c|c|c|c|c|c|}
		\hline
		$v_0=1.5$   & 1 & 2 & 3 & 4 & 5 \\
		\hline
		FD & 9.163e-05 & 2.435e-04 & 4.833e-04 & 8.160e-04 & 1.243e-03 \\
		\hline
		FE & 4.431e-04 & 7.924e-04 & 1.516e-03 & 3.103e-03 & 8.001e-03  \\
		\hline
	\end{tabular}
	\label{table:eigenvalue-error-fixed-open}
\end{table}

\begin{figure}[H]
	\centering
	\begin{subfigure}{0.5\textwidth}
		\includegraphics[width=\linewidth]{../../thesis/img/numerical-experiments/fixed-open/constant-v-v0=0.5}
		\caption{All modes are stable.}
	\end{subfigure}%
	\begin{subfigure}{0.5\textwidth}
		\includegraphics[width=\linewidth]{../../thesis/img/numerical-experiments/fixed-open/constant-v-v0=1.5}
		\caption{All modes are unstable.}
	\end{subfigure}
	\caption{Showing the first 5 eigenvalues of each method. Finite-difference method has much better accuracy than finite-element method.}
	\label{fig:constant-v-fixed-open}
\end{figure}


\section{Subsonic Case}
\begin{figure} [H]
  \centering
  \begin{subfigure}{0.45\textwidth}
    \centering
    \includegraphics[width=\linewidth]{../../thesis/img/numerical-experiments/fixed-fixed/subsonic-v}
    \caption{Dirichlet boundary, all modes are stable.}
  \end{subfigure}%
  \begin{subfigure}{0.45\textwidth}
    \includegraphics[width=\linewidth]{../../thesis/img/numerical-experiments/fixed-open/subsonic-v}
    \caption{The ground mode is unstable, other modes are stable.}
  \end{subfigure}
  \caption{Showing the first 5 modes. It suggests that the subsonic flow in magnetic nozzle is stable.}
\end{figure}

\section{Supersonic Case}
\begin{figure} [H]
  \centering
  \begin{subfigure}{0.45\textwidth}
    \centering
    \includegraphics[width=\linewidth]{../../thesis/img/numerical-experiments/fixed-fixed/supersonic-v}
    \caption{Dirichlet boundary, filtered modes are stable.}
  \end{subfigure}%
  \begin{subfigure}{0.45\textwidth}
    \centering
    \includegraphics[width=\linewidth]{../../thesis/img/numerical-experiments/fixed-open/supersonic-v}
    \caption{Fixed-open boundary, all modes are unstable.}
  \end{subfigure}
  \caption{This suggests that the supersonic flow is stable if the boundary is Dirichlet and unstable if the boundary is left-fixed-right-open.}
\end{figure}


\section{Accelerating Case}
Starting from the singular point, we shoot the solution to the left boundary. We find the set of eigenvalues such that $\tilde{v}(-1)=0$. With these eigenvalues, we can extend the solution to the supersonic region $(0,1]$. The first five eigenvalues are drawn in the graph.
\begin{figure} [H]
	\centering
	\includegraphics[width=0.7\linewidth]{../../thesis/img/numerical-experiments/accelerating-v}
	\caption{All modes are stable.}
	\label{fig:accelerating-v}
\end{figure}

\chapter{Conclusion}
\section{Conclusion}

\section{Summary}

\section{Future Work ?}


\bibliographystyle{plain}
\bibliography{references} 

\include{appendix}

\end{document}
