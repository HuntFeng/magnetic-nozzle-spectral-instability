\chapter{Introduction}
\cite{aikawa_stability_1979}
\section{Plasma}


In this thesis we study the instability of plasma flow in magnetic mirror configuration. We start the thesis by introducing the concept of plasma.

Plasma is one of the four fundamental states of matter, along with solids, liquids, and gases. It is often referred to as the fourth state of matter. Plasma is an ionized gas that consists of highly energized particles, including positively charged ions and negatively charged electrons.

[[Put a figure made by professor Andrei!!!!!!!!!!!!!!!!!]]


In a plasma, the atoms or molecules have been stripped of their electrons, resulting in a collection of charged particles. This ionization process occurs when a gas is subjected to extremely high temperatures or strong electromagnetic fields, which supply sufficient energy to overcome the electrostatic forces that hold electrons in their orbits around atomic nuclei.

Plasma is known for its unique properties. It is an excellent conductor of electricity and is strongly influenced by electromagnetic fields. Plasma also emits light, and examples of natural plasma include stars, such as our Sun, and lightning. Artificially generated plasma can be found in fluorescent lights, plasma televisions, and certain types of industrial torches.

In addition to these applications, plasma has various scientific and technological uses. It is used in plasma physics research, nuclear fusion experiments, plasma cutting and welding, plasma medicine for treating diseases, and even in spacecraft propulsion systems.

Overall, plasma is an intriguing and versatile state of matter with significant implications in various fields of science, technology, and industry.

\section{Magnetic Mirror Configuration}
In this thesis, we are going to deal with plasma flow in magnetic mirror configuration.
The magnetic mirror configuration is a concept in plasma physics and magnetic confinement fusion research. It involves using magnetic fields to confine and control the movement of charged particles, typically in a linear or cylindrical geometry.

In a magnetic mirror configuration, a combination of magnetic field lines is employed to create regions of high magnetic field strength called "magnetic mirrors" or "mirror cells." These magnetic mirrors can trap and reflect charged particles, such as ions or electrons, preventing them from escaping along the magnetic field lines. The particles are confined within the magnetic mirrors, bouncing back and forth between the mirror cells.

The concept of a magnetic mirror has connections to both the magnetic nozzle and solar wind:

1. Magnetic Nozzle: A magnetic nozzle is a device that uses a magnetic field to shape and control the flow of charged particles in a plasma propulsion system. By employing magnetic mirrors, the magnetic nozzle can efficiently direct and accelerate the plasma particles, generating thrust for propulsion. The magnetic field in the nozzle helps collimate and focus the plasma exhaust, increasing its velocity and enhancing the performance of the propulsion system.

2. Solar Wind: The solar wind is a stream of charged particles, primarily electrons and protons, flowing outward from the Sun. The interaction between the solar wind and the Earth's magnetic field is of particular importance. The Earth's magnetic field acts as a barrier, deflecting and trapping the charged particles from the solar wind. This trapping effect is similar to the magnetic mirror configuration, where the Earth's magnetic field lines create mirror cells that confine and redirect the solar wind particles, forming the Van Allen radiation belts.

In summary, the magnetic mirror configuration is a technique that uses magnetic fields to confine and control charged particles. It finds applications in plasma propulsion systems through the use of a magnetic nozzle. Moreover, the concept of magnetic mirrors is relevant to the study of the interaction between the solar wind and the Earth's magnetic field, as it helps in understanding the trapping and confinement of charged particles in the Van Allen radiation belts.

\section{Instability of Plasma Flow}
Since we are going to study the instability of the plasma flow in magnetic mirror configuration. We need to understand the concept of plasma instability.
The instability of plasma flow refers to the tendency of a plasma system to deviate from a stable, equilibrium state and exhibit perturbations or fluctuations in its behavior. These instabilities can arise from various factors, such as the interaction of particles with electromagnetic fields, collective effects, or the presence of gradients in plasma parameters.

Understanding and studying plasma instabilities are crucial for several reasons:

1. Energy Transport: Plasma instabilities can play a significant role in the transport of energy within a plasma system. They can enhance or hinder the transfer of energy between particles, affecting the overall efficiency and behavior of plasma devices. By studying these instabilities, scientists and engineers can gain insights into the mechanisms governing energy transport in plasmas and develop strategies to control and mitigate them.

2. Plasma Confinement: In applications such as magnetic confinement fusion, achieving and maintaining a high degree of plasma confinement is essential for sustained fusion reactions. Instabilities can lead to the loss of plasma particles, reduction in confinement time, and decreased overall plasma performance. By understanding the nature of these instabilities, researchers can design improved confinement strategies and develop techniques to suppress or stabilize them.

3. Plasma Heating: Instabilities can also influence the heating mechanisms in a plasma system. For example, in magnetic fusion devices, instabilities like the ion temperature gradient (ITG) or electron temperature gradient (ETG) instabilities can hinder efficient heating of the plasma. Understanding these instabilities helps in optimizing heating schemes and improving the overall heating efficiency of plasmas.

4. Plasma Diagnostics: Instabilities can manifest as measurable fluctuations in plasma parameters such as density, temperature, and electromagnetic fields. By studying these fluctuations and their characteristics, scientists can employ diagnostic techniques to gain valuable information about the plasma state, identify the presence of instabilities, and assess the stability and health of plasma devices.

Overall, the study of plasma instabilities is crucial for advancing plasma physics research, optimizing plasma devices, and improving our ability to control and utilize plasmas effectively in various applications such as fusion energy, plasma propulsion, materials processing, and astrophysics.

\section{Goals of this Thesis}
The major goal of this thesis is to study the instability of plasma flow in magnetic mirror configuration with different boundary conditions.

Fluid model of plasma will be reviewed and linearized governing equations will be derived in chapter \ref{chap:governing-equations}. The problem will be then formulated as an eigenvalue problem.

In chapter \ref{chap:methodology}, spectral method and shooting method for solving eigenvalue problem will be introduced. 
In the section of spectral method, different discretizations of the operators, such as finite difference and spectral method will be discussed.
Moreover, spectral pollution and its filtering will as also be investigated.

Then in the next section, we will formulate the problem to the form suitable for applying shooting method. We will apply both shooting method and spectral method to the problem.
By comparing the results from two different methods, the credibility of the results are increased.

In chapter \ref{chap:numerical-experiments}, we ill use the method developed in chapter \ref{chap:methodology} to conduct numerical experiments. The goal is to extract the eigenvalues (frequency) of each oscillating mode. 

Conclusion will in chapter \ref{chap:conclusion}.