\documentclass{article}
\usepackage[margin=1in]{geometry}
\usepackage{amsmath, amssymb, amsthm}
\usepackage{physics}
\usepackage{float, subcaption, graphicx}

\title{Magnetic Nozzle - Filtering Spurious Modes Ny Convergen Test}
\author{Hunt Feng}
\date{\today}
\begin{document}
    \maketitle
    
    \section*{Problem}
    The polynomial eigenvalue problem is given by
    \begin{multline} \label{eq:polynomial-eigenvalue-problem}
        \omega^2 v 
        + 2i\omega\left(v_0\pdv{}{z} + \pdv{v_0}{z}\right) v \\
        + \left[ (1-v_0^2)\pdv[2]{}{z}
        -\left(3v_0 + \frac{1}{v_0}\right)\pdv{v_0}{z}\pdv{}{z} 
        - \left(1-\frac{1}{v_0^2}\right)\left(\pdv{v_0}{z}\right)^2 
        - \left(v_0+\frac{1}{v_0}\right)\pdv[2]{v_0}{z} \right]v
        = 0
    \end{multline}

    This problem can be transformed to an eigenvalue problem
    \begin{equation}
        \mqty[ O & I\\A_{21} & A_{22} ]\mqty[ v\\ \omega v] = \omega\mqty[ v\\ \omega v]
        \label{eq:eigenvalue-problem}
    \end{equation}
    where 
    \begin{align*}
        A_{21} &= -\left[ (1-v_0^2)\pdv[2]{}{z}
        -\left(3v_0 + \frac{1}{v_0}\right)\pdv{v_0}{z}\pdv{}{z} 
        - \left(1-\frac{1}{v_0^2}\right)\left(\pdv{v_0}{z}\right)^2 
        - \left(v_0+\frac{1}{v_0}\right)\pdv[2]{v_0}{z} \right] \\
        A_{22} &= -2i\left(v_0\pdv{}{z} + \pdv{v_0}{z}\right)
    \end{align*}

    If we try to investigate the eigenvalues of the problem, the spectral pollution will generate spurious modes.

    \newpage
    \section*{Convergence Test}
    To filter the eigenvalues, we compute the eigenvalues in different resolutions, and only trust those that are converging.

    \subsection*{Subsonic case}
    \begin{figure}[H]
        \begin{subfigure}[b]{0.5\textwidth}
            \includegraphics*[width=0.9\textwidth]{img/eigvals-Mm=0.5.png}
            \caption{No spurious modes in subsonic case.}
        \end{subfigure}%
        \begin{subfigure}[b]{0.5\textwidth}
            \includegraphics*[width=0.9\textwidth]{img/eigfuncs-Mm=0.5.png}
            \caption{First three eigenfunctions.}
        \end{subfigure}
    \end{figure}

    \subsection*{Supersonic case}
    \begin{figure}[H]
        \centering
        \begin{subfigure}[b]{0.5\textwidth}
            \includegraphics*[width=0.9\textwidth]{img/eigvals-Mm=1.5.png}
            \caption{Eigenvalues before filtering.}
        \end{subfigure}%
        \begin{subfigure}[b]{0.5\textwidth}
            \includegraphics*[width=0.9\textwidth]{img/eigfuncs-Mm=1.5.png}
            \caption{Spurious eigenfunctions has weird shape.}
        \end{subfigure}
        \begin{subfigure}[b]{0.5\textwidth}
            \includegraphics*[width=0.9\textwidth]{img/eigvals-Mm=1.5-filtered.png}
            \caption{Only these eigenvalues are converging.}
        \end{subfigure}%
        \begin{subfigure}[b]{0.5\textwidth}
            \includegraphics*[width=0.9\textwidth]{img/eigfuncs-Mm=1.5-filtered.png}
            \caption{Filtered eigenfunctions.}
        \end{subfigure}
    \end{figure}

    \subsection*{Accelerating case}
    \begin{figure}[H]
        \centering
        \begin{subfigure}[b]{0.5\textwidth}
            \includegraphics*[width=0.9\textwidth]{img/eigvals-accelerating.png}
            \caption{Eigenvalues before filtering.}
        \end{subfigure}%
        \begin{subfigure}[b]{0.5\textwidth}
            \includegraphics*[width=0.9\textwidth]{img/eigfuncs-accelerating.png}
            \caption{Spurious eigenfunctions has weird shape.}
        \end{subfigure}
        \begin{subfigure}[b]{0.5\textwidth}
            \includegraphics*[width=0.9\textwidth]{img/eigvals-accelerating-filtered.png}
            \caption{Only these eigenvalues are converging.}
        \end{subfigure}%
        \begin{subfigure}[b]{0.5\textwidth}
            \includegraphics*[width=0.9\textwidth]{img/eigfuncs-accelerating-filtered.png}
            \caption{Filtered eigenfunctions.}
        \end{subfigure}
    \end{figure}
    
    \subsection*{Decelerating case}
    \begin{figure}[H]
        \centering
        \begin{subfigure}[b]{0.5\textwidth}
            \includegraphics*[width=0.9\textwidth]{img/eigvals-decelerating.png}
            \caption{Eigenvalues before filtering.}
        \end{subfigure}%
        \begin{subfigure}[b]{0.5\textwidth}
            \includegraphics*[width=0.9\textwidth]{img/eigfuncs-decelerating.png}
            \caption{Spurious eigenfunctions has weird shape.}
        \end{subfigure}
        \begin{subfigure}[b]{0.5\textwidth}
            \includegraphics*[width=0.9\textwidth]{img/eigvals-decelerating-filtered.png}
            \caption{Only these eigenvalues are converging.}
        \end{subfigure}%
        \begin{subfigure}[b]{0.5\textwidth}
            \includegraphics*[width=0.9\textwidth]{img/eigfuncs-decelerating-filtered.png}
            \caption{Filtered eigenfunctions.}
        \end{subfigure}
    \end{figure}
\end{document}