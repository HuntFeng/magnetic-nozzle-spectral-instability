\chapter{Numerical Experiments}
In this chapter, we will solve the eigenvalue problem, Eq.(\ref{eq:eigenvalue-problem}), with different discretizations. There will be three major categories of methods used. Finite difference (FD) method, finite element (FE) method and spectral element method (SE).

The finite difference method will be used together with equally spaced nodes. The finite element method will use B-spline as basis functions. Finally, the spectral element method uses sine functions as the spectral elements.

The parameters of different discretizations are listed below
\begin{table} [H]
	\centering
	\caption{All methods use 101 nodes in the region $[0,1]$. For FE and SE methods, their accuracy mainly depend on the number of basis functions used, so set the same number of basis for them.}
	\begin{tabular}{|c|c|c|c|}
		\hline
		& FD & FE\_BSPLINE & SE\_SINE  \\
		\hline
		N & 101 & 101 & 101 \\
		\hline
		NUM\_BASIS &  & 50 & 50 \\
		\hline
	\end{tabular}
	\label{table:parameters}
\end{table}

\section{Constant Velocity Case}
Because the existence of exact solution to problems Eq.(\ref{eq:constant_v_problem_dirichlet}). The case with constant velocity profile is used as a sanity check. It allows us to verify the correctness of each method's implementation. This also serves as a reference to the accuracy spectral methods can achieve.

From Fig.(\ref{fig:constant_v}), we see that the order of growth rates obtained by different methods is about $~10^{-14}$ for subsonic case and  $\sim 10^{-12}$ for supersonic case. We will use these numbers as a reference to the accuracy of our numerical methods. If a method produces growth rates with order close to or smaller than these numbers, we consider the growth rates to be 0.

\begin{table} [H]
	\centering
	\caption{Relative error of each eigenvalue.}
	\begin{tabular}{|c|c|c|c|c|c|}
		\hline
		$v_0=0.5$   & 1 & 2 & 3 & 4 & 5 \\
		\hline
		FD & 2.8274e-05 & 1.130e-04 & 2.541e-04 & 4.512e-04 & 7.040e-04 \\
		\hline
		FE & 0.003 & 0.059 & 0.330 & 0.085 & 0.025  \\
		\hline
		SE & 2.896e-05 & 1.157e-04 & 2.603e-04 & 4.626e-04 & 7.217e-04 \\
		\hline
	\end{tabular}
	\begin{tabular}{|c|c|c|c|c|c|}
		\hline
		$v_0=1.5$   & 1 & 2 & 3 & 4 & 5 \\
		\hline
		FD & 0.001 & 0.005 & 0.010 & 0.019 & 0.030 \\
		\hline
		FE & 0.009 & nan & 0.027 & 0.018 & 0.003  \\
		\hline
		SE & 0.001 & 0.005 & 0.011 & 0.019 & 0.030 \\
		\hline
	\end{tabular}
	\label{table:eigenvalue_error}
\end{table}

\begin{figure}[H]
	\centering
	\begin{subfigure}{0.5\textwidth}
		\includegraphics[width=\linewidth]{img/numerical_experiments/constant_v_v0=0.5}
		\caption{All modes are stable.}
	\end{subfigure}%
	\begin{subfigure}{0.5\textwidth}
		\includegraphics[width=\linewidth]{img/numerical_experiments/constant_v_v0=1.5}
		\caption{Filtered modes are stable.}
	\end{subfigure}
	\caption{Showing the first 5 eigenvalues of each method in each case. All methods are close to the exact eigenvalues.}
	\label{fig:constant_v}
\end{figure}


\section{Subsonic Case}
When setting the mid-point velocity to be $M_m=0.5$, we have the subsonic velocity profile. This velocity profile is the orange line shown in Fig.\ref{fig:velocity_profiles}. With Dirichlet boundary condition, $\tilde{v}(\pm 1) =0$. The flow in magnetic nozzle is stable. Fig.\ref{fig:subsonic_v} shows the first few eigenvalues obtained by different discretizations. 

The order of growth rates obtained by different methods is $10^{-13}$, we can consider it to be stable.
\begin{figure} [H]
	\centering
	\includegraphics[width=0.7\linewidth]{img/numerical_experiments/subsonic_v}
	\caption{Showing the first 5 modes. It suggests that the flow in magnetic nozzle with subsonic velocity profile and Dirichlet boundary condition is stable.}
	\label{fig:subsonic_v}
\end{figure}


\section{Supersonic Case}
When the velocity profile is supersonic, shown as purple line in Fig.\ref{fig:velocity_profiles}, spurious modes appeared as predicted in Chap.\ref{chap:theoretical_analysis}. Using the convergence test, we successfully eliminates all unstable modes. Fig.(\ref{fig:supersonic_v}) shows the first few filtered eigenvalues. As we can see the flow is stable.
\begin{figure} [H]
	\centering
	\includegraphics[width=0.7\linewidth]{img/numerical_experiments/supersonic_v}
	\caption{First few filtered eigenvalues are shown. The spurious modes are filtered by convergence test.}
	\label{fig:supersonic_v}
\end{figure}

\section{Accelerating Case}
In this case, the flow starts from subsonic speed and accelerates to supersonic case, the velocity is exactly at sonic point $M_m=1$ at the center of the magnetic nozzle as shown in Fig.(\ref{fig:velocity_profiles})
With Dirichlet boundary condition, spectral method with different discretizations gave negative eigenvalues. This indicates that the perturbation $\tilde{v}$ is damped oscillation, its amplitude will be decrease to zero exponentially in time, $\tilde{v} \sim \exp(\Im(\omega)t)$. Hence, the flow in magnetic nozzle is stable.
\begin{figure} [H]
	\centering
	\includegraphics[width=0.7\linewidth]{img/numerical_experiments/accelerating_v}
	\caption{The filtered eigenvalues are negative. This indicates that the perturbation $\tilde{v}$ is a damped oscillation, their amplitude will decrease to zero as time elapse.}
	\label{fig:accelerating_v}
\end{figure}


\section{Decelerating Case}
As predicted by Proposition.\ref{prop:symmetry_of_eigenvalue}, the growth rates must be positive since they are complex conjugates of the eigenvalues in accelerating case.

\begin{figure} [H]
	\centering
	\includegraphics[width=0.7\linewidth]{img/numerical_experiments/decelerating_v}
	\caption{The flow with decelerating velocity profile and Dirichlet boundaries is unstable. The grow rates are opposite to that in accelerating case.}
	\label{fig:decelerating_v}
\end{figure}

