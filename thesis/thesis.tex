\documentclass[12pt]{report}
\usepackage[margin=2.5cm]{geometry}
\usepackage{amsmath, amssymb, amsthm}
\usepackage{physics}
\usepackage{float, subcaption, graphicx}
\usepackage{hyperref} % comment this in school template
\usepackage{setspace}

\onehalfspacing

\theoremstyle{plain}
\newtheorem{proposition}{Proposition}

\theoremstyle{definition}
\newtheorem{definition}{Definition}

\newtheorem{theorem}{Theorem}

\newtheorem*{remark}{Remark}


\title{Instability of Flow Magnetic Nozzle}
\author{Hunt Feng}

%%%%%%%%%%%%%%%%%%%%%%%%%%%%%%%%%%%%%%%%%%%%%%%%%%%%%%%%%%%%%%%%
% END OF FRONTMATTER SECTION
%%%%%%%%%%%%%%%%%%%%%%%%%%%%%%%%%%%%%%%%%%%%%%%%%%%%%%%%%%%%%%%%
\begin{document}
% Typeset the title page
\maketitle

\tableofcontents % remove this when using school template
% move abstract before the document in school template
\abstract{
    Spectral theory is a common technique for analyzing the instability of a dynamical system. By discretizing the linearized equations motion of magnetic nozzle, the instability problem becomes an algebraic eigenvalue problem. Given Dirichlet boundary condition, we found that the flow in magnetic nozzle is stable. Different discretizations, such as finite difference, finite element and spectral element method agree with each other. By studying the convergence of different modes, we successfully eliminated the spurious unstable modes occur in supersonic and transonic cases.
}

%%%%%%%%%%%%%%%%%%%%%%%%%%%%%%%%%%%%%%%%%%%%%%%%%%%%%%%%%%%%%%%%
% FIRST CHAPTER OF THESIS BEGINS HERE
%%%%%%%%%%%%%%%%%%%%%%%%%%%%%%%%%%%%%%%%%%%%%%%%%%%%%%%%%%%%%%%%
\chapter{Introduction}
\section{Flow in Magnetic Nozzle}
\subsection{Magnetic Nozzle}
A magnetic nozzle is a device that uses a magnetic field to shape and control the flow of charged particles in a plasma propulsion system, see Fig.\ref{fig:magnetic-nozzle}. By employing the magnetic mirror configuration, the magnetic nozzle can effciently direct and accelerate the plasma flow, generating thrust for propulsion. The magnetic field in the nozzle helps collimate and focus the plasma exhast, increasing its velocity and enhancing the performance of the propulsion system.

\begin{figure}[htbp]
  \centering
  \includegraphics[width=0.7\linewidth]{../../thesis/img/introduction/magnetic-nozzle}
  \caption{A simplified model of the magnetic nozzle}
  \label{fig:magnetic-nozzle}
\end{figure}

\subsection{Magnetic Field in Magnetic Nozzle}
In 1D problem, the magnetic field can be modeled as
\[ B(z) = B_0 \left[1 + R\exp(-\left(\frac{x}{\delta}\right)^2)\right] \]
where $1+R$ is the magnetic mirror ratio, and $\delta$ determines the spread of the magnetic field. It is shown in Fig.(\ref{fig:magnetic-field}).
\begin{figure}[H]
	\centering
	\includegraphics[width=0.7\linewidth]{../../thesis/img/introduction/magnetic-field}
	\caption{This is the magnetic field in nozzle with mirror ratio $1+R=B_{max}/B_{min}=2.5$, and the spread of magnetic field, $\delta=0.1/0.3=0.\bar{3}$. }
	\label{fig:magnetic-field}
\end{figure}

\subsection{Velocity Profile at Equilibrium}
Let $n_0$ and $v_0$ be the density and velocity at equilibrium (stationary solution), we know that $\pdv*{n_0}{t}=0$ and $\pdv*{v_0}{t}=0$, therefore $n_0$ and $v_0$ satisfy the so-called equilibrium condition,
\begin{align*}
	&\pdv{z}(\frac{n_0v_0}{B}) = 0 \\
	&v_0\pdv{v_0}{z} = -c_s^2\frac{1}{n_0}\pdv{n_0}{z} 
\end{align*}

Let $M(z) = v_0(z)/c_s$ be the mach number (nondimensionalized velocity). The equations of motion become
\begin{align*}
	&B\pdv{z}(\frac{n_0M}{B}) = 0\\
	&M\pdv{M}{z} = -\frac{1}{n_0}\pdv{n_0}{z}
\end{align*}
Substitute $\frac{1}{n_0}\pdv*{n_0}{z}$ using first equation, the conservation of momentum becomes
\[ (M^2-1)\pdv{M}{z} = -\frac{M}{B}\pdv{B}{z} \]

Notice that there is a singularity at $M=1$, the sonic speed.

This is a separable equation, integrate it and use the conditions at midpoint $B(0)=B_m, M(0)=M_m$ we get
\[ M^2e^{-M^2} = \frac{B^2}{B_m^2}M_m^2e^{-M_m^2} \]
We can now express $M$ using the Lambert W function,
\[ M(z) = \left[ -W_k\left(-\frac{B(z)^2}{B_m^2}M_m^2e^{-M_m^2}\right) \right]^{1/2} \]
where the subscript $k$ of $W$ stands for branch of Lambert W function. When $k=0$, it is the subsonic branch; When $k=-1$, it is the supersonic branch. Below shows a few cases of the solution.
\begin{itemize}
	\item $M_m < 1, k=0$, subsonic velocity profile.
	\item $M_m = 1$, $k=0$ for $x<0$ and $k=-1$ for $x>0$, accelerating profile
	\item $M_m = 1$, $k=-1$ for $x<0$ and $k=0$ for $x>0$, decelerating profile
	\item $M_m > 1, k=-1$, supersonic velocity profile
\end{itemize}
 Fig.(\ref{fig:velocity_profiles}) shows some cases of the solution.
\begin{figure}[H]
	\centering
	\includegraphics[width=0.7\linewidth]{../../thesis/img/introduction/velocity-profiles}
	\caption{The velocity profile in the magnetic nozzle is completely determined by $M_m$, the velocity at the midpoint, $z=0$. For the transonic velocity profiles, $M_m$ alone is not enough to determine the profile, we need to specify the branch of Lambert W function to determine whether it is accelerating or decelerating.}
	\label{fig:velocity_profiles}
\end{figure}

\subsection{Flow in Similar Configuration: Bondi-Parker Flow}
Bondi derived a steady-state solution for accretion flow which is governed by Bernoulli's equation in sperical symmetry around a point mass in 1952. Then Parker solved a similar problem but with outward wind in 1958. \cite{aikawa_stability_1979,bondi_spherically_1952,keto_stability_2020} The equilibrium velocity profiles in such configuration are shown in Fig.\ref{fig:BP-flow-velocity}.

\begin{figure}[htbp]
    \centering
    \includegraphics[width=0.7\textwidth]{../../thesis/img/introduction/steady-state-BP-flow}
    \caption{Representative trajectories of the steady-state BP flow in non-dimensional units. \cite{keto_stability_2020} The upward pink line represents a outward wind, it accelerates from subsonic to supersonic. The downward pink line represents an accretion flow, it accelerates towards the mass point. The green lines below the pink lines represent subsonic flows, and the green lines above represent supersonic flows. Orange lines are physically impossible scenarios.}
    \label{fig:BP-flow-velocity}
\end{figure}

Solar wind is an example of Bondi-Parker flow. The solar wind is a stream of charged particles, primarily electrons and protons, flowing outward from the Sun. 

\section{Instability of Plasma Flow}
In this section, plasma instability will be introduced and from that we will discuss the importance of this research.

The instability of plasma flow refers to the tendency of a plasma system to deviate from a stable, equilibrium state and exhibit perturbations or fluctuations in its behavior. These instabilities can arise from various factors, such as the interaction of particles with electromagnetic fields, collective effects, or the presence of gradients in plasma parameters.

\begin{figure}[htbp]
	\centering
	\includegraphics[width=0.7\linewidth]{../../thesis/img/introduction/stability-visualization}
	\caption{Mechanical analogy of various types of equilibrium. \cite{chen_introduction_2016}}
	\label{fig:stability-visualization}
\end{figure}

Plasma flow in magnetic mirror configurations have been studied extensively in plasma physics due to its frequent precense in many areas such as the accretion flow \cite{jockers_stability_1968,aikawa_stability_1979}, and magnetic nozzle\cite{smolyakov_quasineutral_2021}. However, the stability of these configurations remains a debatable subject.

\section{Goal of this Thesis}
The goal of this thesis is to study the instabilities of the magnetic mirror configuration given certain boundary conditions and equilibrium velocity profiles.

To achieve the goal, first we need to study the spectral method for solving the instability problem. To use spectral method, it is necessary to understand different discretizations of the operators, such as finite difference, finite element and spectral element method.

Once the spectral method is introduced, we can use it to study the instability of plasma flow in magnetic nozzle as an eigenvalue problem. We can obtain results by using different discretization techniques. By comparing the results from different approach, we can increase the credibility of the true solution.

However, spectral method is not suitable when the equilibrium velocity profile is transonic due to the precense of singularity at the sonic point. We need to solve the singular perturbation problem around the singularity analytically. Then starting from the singular point, we can use shooting method to find eigenvalues.

\section{Outline of the thesis}
The thesis will be divided into several chapters. In chapter 2, spectral method will be introduced. Chapter 3 will focus on the physics of flow in magnetic mirror configuration and derive the governing equations for charged particles, the linearized equations of motions. Following this, we will analyze the problem analytically in chapter 4. Moving to chapter 5, numerical experiments will be conducted. The conclusion will be made in chapter 6.

\chapter{Theoretical Analysis} \label{chap:theoretical-analysis}
In order to determine the best tool to tackle the problem. We need to first simplify the problem.

\section{Fluid Description for Flow}
In this section, we will derive the governing equations of the flow in magnetic nozzle, starting from the fluid description for plasma.

We start by deriving the usefule form of the conservation of density,
\[ \pdv{n}{t} + \div(n\mathbf{v}) = 0 \]
where $\mathbf{v}$ here denotes the fluid velocity of the plasma flow.

We can get the fluid velocity by taking the integral
\[ \mathbf{v} = \frac{1}{n}\int_{\mathbb{R}^3} \mathbf{v}_p f(\mathbf{x}, \mathbf{v}_p, t) d^3\mathbf{v}_p \]

Denote the particle velcity as $\mathbf{v}_p$, we can decompose the particle velocity vector as $\mathbf{v}_p = (v_\parallel,v_\perp)$, where $v_\parallel$ and $v_\perp$ are the magnitudes of components that are parallel to and perpendicular to the magnetic field line, respectively. Due to the Lorentz force, the charged particles gyrates about the magnetic field lines, see Fig.\ref{fig:gyrate-along-b-field}. Hence, the $v_\perp$ will be averaged to zero the expression for plasma fluid velocity can be simplied as
\[\mathbf{v} = v\mathbf{B}/B\]
where $v$ is the fluid speed along the magnetic field lines. This makes sence because the charged particles flows along $\mathbf{B}$.

By expanding the divergence term, and using the divergence free condition $\div B=0$, we have
\[ \pdv{n}{t} + \mathbf{B}\cdot \grad(\frac{nv}{B}) = 0 \]
Since the magnetic field lines are aligned with the central axis of the nozzle, which we denote as z-axis, so $\mathbf{B} = B\hat{z}$. Now we obtain the conservation of density for the magnetic nozzle,
\begin{equation}
    \pdv{n}{t} + B\pdv{z}(\frac{nv}{B}) = 0
\end{equation}

The second governing equation is the conservation of momentum,
\[ mn\pdv{\mathbf{v}}{t} + mn\mathbf{v}\cdot\grad\mathbf{v} = -\grad p \]
where $m$ is the ion mass. This equation tells us the plasma flow is driven by pressure.

The equation of state is given by the isothermal condition,
\begin{equation} \label{eq:eos}
    p = nk_BT
\end{equation}
There are two main reasons. First the plasma particles are confined to the magnetic field lines. This reduces the particle collisions and energy exchanges. Moreover, the electrons have high mobility, they will quickly fill up any charge cavities and thus maintain a constant temperature. Hence, we can safely assume the plasma flow is isothermal.

Therefore, we have
\begin{equation}
    \pdv{v}{t} + v\pdv{v}{z} = -c_s^2\frac{1}{n}\pdv{n}{z}
\end{equation}
where $c_s^2 = k_BT/m$ is the square of sound speed.

Therefore, the dynamics of the plasma flow in magnetic nozzle can be characterized by the conservation of density and momentum,
\begin{align*}
     & \pdv{n}{t} + B\pdv{z}(\frac{nv}{B}) = 0                \\
     & \pdv{v}{t} + v\pdv{v}{z} = -c_s^2\frac{1}{n}\pdv{n}{z}
\end{align*}
The magnetic field profile was discussed in Sec.\ref{sec:magnetic-field-in-nozzle}.

In this research, we are interested in the stability of the equilibrium flow in the nozzle. Let's denote $n_0$ and $v_0$ as equilibrium density and equilibrium velocity, respectively. Since they are stationary (time independent) solutions to the above set of equations, so they satisfy the so-called equilibrium condition,
\begin{align*}
     & B\pdv{z}(\frac{n_0v_0}{B})  = 0                   \\
     & v_0\pdv{v_0}{z} = -c_s^2\frac{1}{n_0}\pdv{n_0}{z}
\end{align*}

\section{Non-dimensionalization}
For convenience, we nondimensionalize the governing equations by normalizing the velocity to $c_s$, $v\mapsto v/c_s$, $z$ to system length $L$, $z \mapsto z/L$ and time $t\mapsto c_s t/L$. The governing equations become
\begin{align}
     & \pdv{n}{t} + n\pdv{v}{z} + v\pdv{n}{z} - nv\frac{\partial_z B}{B} = 0
    \label{eq:conservation-of-density}
    \\
     & n\pdv{v}{t} + nv\pdv{v}{z} = -\pdv{n}{z}
    \label{eq:conservation-of-momentum}
\end{align}
and the nondimensionalized equilibrium condition is
\begin{align}
     & \pdv{z}(\frac{n_0v_0}{B}) = 0 \label{eq:equilibrium-convervation-of-mass}                    \\
     & v_0\pdv{v_0}{z} = -\frac{1}{n_0}\pdv{n_0}{z} \label{eq:equilibrium-convervation-of-momentum}
\end{align}

\section{Velocity Profile at Equilibrium}
Let $n_0$ and $v_0$ be the density and velocity at equilibrium, we know that $\pdv*{n_0}{t}=0$ and $\pdv*{v_0}{t}=0$ for the solution is stationary, in other words time independent. Therefore $n_0$ and $v_0$ satisfy the so-called equilibrium condition,
\begin{align*}
     & \pdv{z}(\frac{n_0v_0}{B}) = 0                     \\
     & v_0\pdv{v_0}{z} = -c_s^2\frac{1}{n_0}\pdv{n_0}{z}
\end{align*}

Let $M(z) = v_0(z)/c_s$ be the mach number (nondimensionalized velocity). The equations of motion become
\begin{align*}
     & B\pdv{z}(\frac{n_0M}{B}) = 0             \\
     & M\pdv{M}{z} = -\frac{1}{n_0}\pdv{n_0}{z}
\end{align*}
Substitute $\frac{1}{n_0}\pdv*{n_0}{z}$ using first equation, the conservation of momentum becomes
\[ (M^2-1)\pdv{M}{z} = -\frac{M}{B}\pdv{B}{z} \]

Notice that there is a singularity at $M=1$, the sonic speed.

This is a separable equation, integrate it and use the conditions at midpoint $B(0)=B_m, M(0)=M_m$ we get
\[ M^2e^{-M^2} = \frac{B^2}{B_m^2}M_m^2e^{-M_m^2} \]
We can now express $M$ using the Lambert W function,
\[ M(z) = \left[ -W_k\left(-\frac{B(z)^2}{B_m^2}M_m^2e^{-M_m^2}\right) \right]^{1/2} \]
where the subscript $k$ of $W$ stands for branch of Lambert W function. When $k=0$, it is the subsonic branch; When $k=-1$, it is the supersonic branch. Below shows a few cases of the solution.
\begin{itemize}
    \item $M_m < 1, k=0$, subsonic velocity profile.
    \item $M_m = 1$, $k=0$ for $x<0$ and $k=-1$ for $x>0$, accelerating profile
    \item $M_m = 1$, $k=-1$ for $x<0$ and $k=0$ for $x>0$, decelerating profile
    \item $M_m > 1, k=-1$, supersonic velocity profile
\end{itemize}
Fig.(\ref{fig:velocity-profiles}) shows some cases of the solution.
\begin{figure}[H]
    \centering
    \includegraphics[width=0.7\linewidth]{img/velocity-profiles}
    \caption{The velocity profile in the magnetic nozzle is completely determined by $M_m$, the velocity at the midpoint, $z=0$. For the transonic velocity profiles, $M_m$ alone is not enough to determine the profile, we need to specify the branch of Lambert W function to determine whether it is accelerating or decelerating.}
    \label{fig:velocity-profiles}
\end{figure}

\section{Linearized Governing Equations}
As illustrated in Sec.\ref{sec:two-stream-instability}, it is essential to linearize the governing equations in order to investigate the instability of plasma. Now we are going to derive the linearized governing equations with the equilibrium conditions given in above.

Let $n = n_0(z) + \tilde{n}(z,t)$ and $v = v_0(z) + \tilde{v}(z,t)$, where $\tilde{n}$ and $\tilde{v}$ are small perturbed quantities.

We first linearize Eq.(\ref{eq:conservation-of-density}) by setting $n=n_0+\tilde{n}$ and $v=v_0+\tilde{v}$,
\[    \pdv{(n_0+\tilde{n})}{t}
    + (n_0+\tilde{n})\pdv{(v_0+\tilde{v})}{z}
    + (v_0+\tilde{v})\pdv{(n_0+\tilde{n})}{z}
    - (n_0+\tilde{n})(v_0+\tilde{v})\frac{\partial_z B}{B} = 0
\]
By ignoring the second order perturbations, we obtain
\[ \frac{1}{n_0}\pdv{\tilde{n}}{t}
    + \pdv{v_0}{z} + \frac{\tilde{n}}{n_0}\pdv{v_0}{z} + \pdv{\tilde{v}}{z}
    + \frac{v_0}{n_0}\pdv{n_0}{z} + \frac{\tilde{v}}{n_0}\pdv{n_0}{z} + \frac{v_0}{n_0}\pdv{\tilde{n}}{z}
    - v_0\frac{\partial_z B}{B} - \tilde{v}\frac{\partial_z B}{B} - \tilde{n}\frac{v_0}{n_0}\frac{\partial_z B}{B} = 0
\]


Using the equilibrium condition Eq.(\ref{eq:equilibrium-convervation-of-mass}), some of the terms are canceled. Moreover, the last term can be written as
\[ \tilde{n}\frac{v_0}{n_0}\frac{\partial_z B}{B} = \frac{\tilde{n}}{n_0}\left( \frac{\partial_z n_0}{n_0}v_0 + \pdv{v_0}{z} \right) \]
Now, we get the linearized conservation of mass,
\begin{equation} \label{eq:linearized-conservation-of-density}
    \frac{1}{n_0}\pdv{\tilde{n}}{t}
    + \pdv{\tilde{v}}{z} + v_0\tilde{Y} + \tilde{v}\frac{\partial_z n_0}{n_0} - \tilde{v}\frac{\partial_z B}{B} = 0
\end{equation}
where
\[ \tilde{Y} \equiv \frac{1}{n_0}\pdv{\tilde{n}}{z} - \frac{\partial_z n_0}{n_0^2}\tilde{n} = \pdv{z}(\frac{\tilde{n}}{n_0}) \]

To linearize the conservation of momentum, we follow the same logic by substituting $n=n_0+\tilde{n}$, and $v=v_0+\tilde{v}$ in Eq.(\ref{eq:conservation-of-momentum}),
\[ (n_0+\tilde{n})\pdv{(v_0+\tilde{v})}{t} + (n_0+\tilde{n})(v_0+\tilde{v})\pdv{(v_0+\tilde{v})}{z} = -\pdv{n}{z} \]

Again, ignore second order perturbations and rearange terms, we have
\[ \pdv{v_0}{t} + v_0\pdv{v_0}{z} + \tilde{v}\pdv{v_0}{z}
    = -\frac{1}{n_0}\pdv{n_0}{z} -\frac{1}{n_0}\pdv{\tilde{n}}{z} -v_0\frac{v_0}{z} - \frac{\tilde{n}}{n_0}v_0\pdv{v_0}{z} \]
Using the equilibrium condition Eq.(\ref{eq:equilibrium-convervation-of-momentum}) on the RHS, we get the linearized conservation of momentum,
\begin{equation} \label{eq:linearized-conservation-of-momentum}
    \pdv{\tilde{v}}{t} + \pdv{(v_0\tilde{v})}{z} = -\tilde{Y}
\end{equation}

\section{Polynomial Eigenvalue Problem}
We can further simplify the problem by combining Eq.(\ref{eq:linearized-conservation-of-density}) and Eq.(\ref{eq:linearized-conservation-of-momentum}) into a single equation. We can substitute Eq.(\ref{eq:linearized-conservation-of-momentum}) into Eq.(\ref{eq:linearized-conservation-of-density}) to eliminate $\tilde{Y}$,

\begin{equation} \label{eq:single-governing-equation}
    \pdv{t}\frac{\tilde{n}}{n_0}
    + \pdv{\tilde{v}}{z} - v_0\left(\pdv{t}\tilde{v}
    + \pdv{(v_0\tilde{v})}{z}\right)
    + \tilde{v}\frac{\partial_z n_0}{n_0}
    - \tilde{v}\frac{\partial_z B}{B}
    = 0
\end{equation}

In order to investigate the instability of the flow, we need formulate it as an eigenvalue problem. To do that, we assume the perturbed density and velocity are oscillatory, i.e. $\tilde{n}, \tilde{v} \sim \exp(-i\omega t)$, where $\omega$ is the oscillation frequency of the perturbed quantities. This frequency can be a complex number.

As illustrated in Sec.\ref{sec:two-stream-instability}, the flow can be stable or unstable depending on the imaginary part of the frequency. If $\Im(\omega) > 0$, then the perturbed quantities $\tilde{n} \sim \exp(\Im(\omega) t)$, which means it grows exponentially with time, hence unstable. If $\Im(\omega) \leq 0$, then the amplitude of the perturbed quanties are either unchanged or exponentially decreasing, hence the flow is stable.

By assuming oscillatory perturbed quantities, Eq.(\ref{eq:single-governing-equation}) becomes,
\begin{equation}
    -i\omega\frac{\tilde{n}}{n_0}
    + \pdv{\tilde{v}}{z} - v_0\left(-i\omega\tilde{v}
    + \pdv{(v_0\tilde{v})}{z}\right)
    + \tilde{v}\frac{\partial_z n_0}{n_0}
    - \tilde{v}\frac{\partial_z B}{B}
    = 0
\end{equation}

Using the equilibrium condition Eq.(\ref{eq:equilibrium-convervation-of-mass}), we can eliminate the term $\partial_z B/B$,
\[
    -i\omega\frac{\tilde{n}}{n_0}
    + \pdv{\tilde{v}}{z}
    + v_0\left(i\omega \tilde{v} - v_0\pdv{\tilde{v}}{z} - \tilde{v}\pdv{v_0}{z} \right)
    - \tilde{v}\frac{\partial_z v_0}{v_0}
    = 0
\]

Rearrange terms, we have
\[
    -i\omega\frac{\tilde{n}}{n_0}
    + i\omega v_0\tilde{v}
    + (1-v_0^2)\pdv{\tilde{v}}{z}
    - \left(v_0+\frac{1}{v_0}\right)\pdv{v_0}{z}\tilde{v} = 0
\]

Now we take $\pdv*{t}$ on Eq.(\ref{eq:linearized-conservation-of-momentum}). Recall the fact that $\tilde{Y} = \partial_z(\tilde{n}/n_0)$, we have
\[
    \omega^2\tilde{v} + i\omega\left(v_0\pdv{\tilde{v}}{z} + \tilde{v}\pdv{v_0}{z}\right)
    = \pdv{t}\pdv{z}(\frac{\tilde{n}}{n_0})
\]
Apply $\partial_t$ operator first, we get
\[
    \omega^2\tilde{v} + i\omega\left(v_0\pdv{\tilde{v}}{z} + \tilde{v}\pdv{v_0}{fz}\right)
    = \pdv{z}(-i\omega v_0\tilde{v}
    - (1-v_0^2)\pdv{\tilde{v}}{z}
    + \left(v_0+\frac{1}{v_0}\right)\pdv{v_0}{z}\tilde{v})
\]
Expand the RHS and collect terms, we get
\begin{equation} \label{eq:polynomial-eigenvalue-problem}
    \begin{aligned}
         & \omega^2 \tilde{v}                                          \\
         & +2i\omega\left(v_0\pdv{}{z} + \pdv{v_0}{z}\right) \tilde{v} \\
         & +\left[ (1-v_0^2)\pdv[2]{}{z}
            -\left(3v_0 + \frac{1}{v_0}\right)\pdv{v_0}{z}\pdv{}{z}
            - \left(1-\frac{1}{v_0^2}\right)\left(\pdv{v_0}{z}\right)^2
            - \left(v_0+\frac{1}{v_0}\right)\pdv[2]{v_0}{z} \right]\tilde{v}
        = 0
    \end{aligned}
\end{equation}

In mathematical terms, Eq.(\ref{eq:polynomial-eigenvalue-problem}) is a polynomial eigenvalue problem, where $\omega$ is an eigenvalue to the problem, and the velocity perturbation $\tilde{v}$ is an eigenfunction associated with the eigenvalue $\omega$. In this chapter we will discuss the methods to tackle this problem.

\section{Analytical Solution to Constant Velocity Case}
\subsection{Dirichlet Boundary}
If we set the velocity profile of the equilibrium flow to constant $v_0=\text{const}$, then Eq.(\ref{eq:polynomial-eigenvalue-problem}) becomes a simple boundary value problem with second order constant coefficients differential equation.

\begin{equation} \label{eq:constant-v-problem-dirichlet}
    \omega^2\tilde{v} + 2i\omega v_0\pdv{\tilde{v}}{z} + (1-v_0^2)\pdv[2]{\tilde{v}}{z} = 0
    \quad
    \tilde{v}(-1) = \tilde{v}(1) = 0
\end{equation}

The solution to this problem is
\begin{equation} \label{eq:constant-v-solution-dirichlet}
    \tilde{v} = C\left[ \exp\left(i\omega\frac{z+1}{v_0+1}\right) - \exp\left(i\omega\frac{z+1}{v_0-1}\right) \right]
\end{equation}
where $C\in\mathbb{C}$ is a complex constant, and the frequencies are $\omega=n\pi(1-v_0^2)/2$ with $n\in\mathbb{Z}$.

This result tells us for constant velocity case, the flow in magnetic nozzle is stable regardless the velocity $v_0$. It is worth to mention $v_0=1$ is a singular point of this problem.

We will use this to benchmark the simulation results.

\begin{figure}[H]
    \centering
    \begin{subfigure}{0.5\textwidth}
        \includegraphics[width=\linewidth]{img/exact-fixed-fixed-v0=0.5}
        \caption{Subsonic}
    \end{subfigure}%
    \begin{subfigure}{0.5\textwidth}
        \includegraphics[width=\linewidth]{img/exact-fixed-fixed-v0=1.5}
        \caption{Supersonic}
    \end{subfigure}
    \caption{The plots show the first three non-zero exact solutions to Eq.(\ref{eq:constant-v-problem-dirichlet}) for both subsonic and supersonic case. These solutions are stable.}
    \label{fig:exact-v-dirichlet}
\end{figure}

\subsection{Fixed-Open Boundary}

\begin{equation} \label{eq:constant-v-problem-fixed-open}
    \omega^2\tilde{v} + 2i\omega v_0\pdv{\tilde{v}}{z} + (1-v_0^2)\pdv[2]{\tilde{v}}{z} = 0
    \quad
    \tilde{v}(-1) = \pdv{\tilde{v}}{z}\,(1) = 0
\end{equation}

The solution to this problem is
\begin{equation} \label{eq:constant-v-solution-fixed-open}
    \tilde{v} = C \left[\exp\left(i\omega\frac{z+1}{v_0+1}\right)
        - \exp\left(i\omega\frac{z+1}{v_0-1}\right) \right]
\end{equation}
where $C\in\mathbb{C}$, and $\omega = (v_0^2 - 1) \left[\frac{n\pi}{2} - \frac{1}{4}i\ln(\frac{v_0-1}{v_0+1})\right]$ with $n\in\mathbb{Z}$. The term $i\ln((v_0-1)/(v_0+1))\in\mathbb{C}$ and its imaginary part is positive for any $v_0\neq 1$. Therefore,
\begin{itemize}
    \item If $v_0<1$, then $\Im(\omega)<0$, it's damped oscillation, hence stable.
    \item If $v_0>1$, then $\Im(\omega)>0$, it's unstable.
\end{itemize}

Worth to mention, this is a very interesting solution with the following properties,
\begin{enumerate}
    \item The growth rate is independent the mode number $n$.
    \item The ground mode $n=0$ for subsonic case has non-zero real part and imaginary part.
\end{enumerate}

\begin{figure}[H]
    \centering
    \begin{subfigure}{0.5\textwidth}
        \includegraphics[width=\linewidth]{img/exact-fixed-open-v0=0.5}
        \caption{Subsonic, stable flow.}
    \end{subfigure}%
    \begin{subfigure}{0.5\textwidth}
        \includegraphics[width=\linewidth]{img/exact-fixed-open-v0=1.5}
        \caption{Supersonic, unstable flow.}
    \end{subfigure}
    \caption{The plots show the first three exact solutions to Eq.(\ref{eq:constant-v-problem-fixed-open}) for both subsonic and supersonic case. The flow is stable for subsonic case and unstable for supersonic case.}
    \label{fig:exact-v-fixed-open}
\end{figure}


\chapter{Spectral Method and Spectral Pollution}
\section{Spectral Method}
Spectral method is one of the best tools to solve PDE and ODE problems. \cite{trefethen_spectral_2000} The central idea of spectral method is by discretizing the equation, we can transform that to a linear system or an eigenvalue problem.

Here we reformulate the polynomial eigenvalue problem, Eq.(\ref{eq:polynomial-eigenvalue-problem}) as the following, 
\begin{equation} \label{eq:eigenvalue-problem}
	\mqty[ 0 & 1\\ \hat{M} & \hat{N} ]\mqty[ \tilde{v}\\ \omega \tilde{v}] = \omega\mqty[ \tilde{v}\\ \omega \tilde{v}]
\end{equation}
where the operators $\hat{M}$ and $\hat{N}$ are defined as
\begin{align*}
	\hat{M} &= -\left[(1-v_0^2)\pdv[2]{}{z} 
	-\left(3v_0 + \frac{1}{v_0}\right)\pdv{v_0}{z}\pdv{}{z} 
	- \left(1-\frac{1}{v_0^2}\right)\left(\pdv{v_0}{z}\right)^2 
	- \left(v_0+\frac{1}{v_0}\right)\pdv[2]{v_0}{z}\right] \\
	\hat{N} &= -2i\left(v_0\pdv{}{z} +\pdv{v_0}{z} \right) 
\end{align*}
This becomes an ordinary algebraic eigenvalue problem if we discretize the operators and the function $\tilde{v}$. In this thesis, finite-difference, finite-element and spectral-element discretizations are used.

\section{Spectral Pollution and Spurious Modes}
In this section, we will discuss an important phenomenon we observe throughout the numerical experiments. It is the phenomenon called spectral pollution. Then we will provide a method to filter these spurious modes.

Spectral pollution refers to the phenomenon which some eigenvalues are not converging to the correct value when the mesh density is increased. When solving eigenvalue problems using spectral methods with finite difference or finite element approximations, spectral pollution might occur. \cite{llobet_spectral_1990}

\subsection{Illustration: Constant velocity case}
We will illustrate the spectral pollution by solving Eq.(\ref{eq:polynomial-eigenvalue-problem}) with constant velocity profile, $v_0=\text{const}$, using spectral method with finite-difference method.

Let $v_0\neq 1$ be a constant, then Eq.(\ref{eq:polynomial-eigenvalue-problem}) becomes 
\begin{equation} \label{eq:constant-v-problem-dirichlet}
  \omega^2\tilde{v} + 2i\omega v_0\pdv{\tilde{v}}{z} + (1-v_0^2)\pdv[2]{\tilde{v}}{z} = 0
\end{equation}

The dispersion relation can be obtained by substituting $\tilde{v} \sim \exp(-i\omega t + kx)$ into Eq.(\ref{eq:constant-v-problem-dirichlet}),
\begin{equation} \label{dispersion-relation}
	\omega = k(v_0 \pm 1) 
\end{equation}
where $k$ is the wave number. We see that all modes should be stable for $\omega\in\mathbb{R}$.

\begin{figure}[htbp]	
  \centering
	\begin{subfigure}[b]{0.4\linewidth}
		\includegraphics[width=\linewidth]{figures/eigvals-bad} 
		\caption{Unfiltered eigenvalues.}
	\end{subfigure}%
	\begin{subfigure}[b]{0.6\linewidth}
		\includegraphics[width=\linewidth]{figures/eigvecs-bad} 
		\caption{First few unfiltered eigenfunctions.}
	\end{subfigure}
  \caption{Spurious modes occurs when solving Eq.(\ref{eq:constant-v-problem-dirichlet}). Finite difference discretization was used. A considerably high resolution mesh with 501 points was used. In fact, spurious modes occurs regardless of the resolution.}
	\label{fig:results-bad}
\end{figure}

A good way to filter the spurious modes is by doing a convergence test, see Fig.(\ref{fig:converging-test}). Since the eigenvalues, Eq.(\ref{dispersion-relation-G}), are changing with mesh resolution. We can simply solve the problem using spectral method under different mesh resolution. Then filter out the eigenmodes that are not converging.

\begin{figure}[htbp]
  \begin{center}
    \includegraphics[width=0.7\textwidth]{figures/convergence-test.png}
  \end{center}
  \caption{Converging test works since spurious eigenvalues change under different resolution. We can see that only the eigenvalues on the real axis are not changing dramatically when resolution increases.}
  \label{fig:converging-test}
\end{figure}


\begin{figure}[htbp]
	\centering
	\begin{subfigure}[b]{0.4\linewidth}
		\includegraphics[width=\linewidth]{figures/eigvals-good} 
		\caption{Filtered eigenvalues.}
	\end{subfigure}%
	\begin{subfigure}[b]{0.6\linewidth}
		\includegraphics[width=\linewidth]{figures/eigvecs-good} 
		\caption{First few filtered eigenfunctions.}
	\end{subfigure}
	\caption{Filtered spurious modes by converging test.}
	\label{fig:results-good}
\end{figure}


\section{Singular Perturbation}
\begin{frame}{Existence of Singularity}
  \begin{figure}[htbp]
    \begin{center}
      \includegraphics[width=0.95\textwidth]{figures/results-bad-accelerating-v.png}
    \end{center}
    \caption{Spectral method failed to resolve meaningful eigenfunctions (and eigenvalues) due to the existence of the singularity at $z=0$.}
    \label{fig:bad-accelerating-v}
  \end{figure}
\end{frame}

\begin{frame}{Shooting Method}
  \begin{itemize}
    \item Expand $\tilde{v}$ near the singularity using Frobenius method.
    \item Pick up regular solutions and shoot them to the left.
    \item Eigenvalues are found by matching the Dirichlet BC.
  \end{itemize}
  \begin{figure}[htbp]
    \begin{center}
      \includegraphics[width=0.7\textwidth]{figures/results-accelerating-v.png}
    \end{center}
    \caption{The solutions crosses the singular point smoothly. All modes are stable.}
    \label{fig:good-accelerating-v}
  \end{figure}
\end{frame}

\chapter{Numerical Experiments}
\section{Constant Velocity Case - Subsonic}
\begin{figure}[H]
  \centering
  \begin{subfigure}{0.45\textwidth}
    \includegraphics[width=0.9\linewidth]{../../thesis/img/numerical-experiments/fixed-fixed/constant-v-v0=0.5}
    \caption{Dirichlet boundary, all modes are stable.}
  \end{subfigure}%
  \begin{subfigure}{0.45\textwidth}
    \includegraphics[width=\linewidth]{../../thesis/img/numerical-experiments/fixed-open/constant-v-v0=0.5}
    \caption{Fixed-open boundary, all modes are stable.}
  \end{subfigure}
  \caption{Showing the first 5 eigenvalues. In the Dirichlet boundary case, all methods are close to the exact eigenvalues. Meanwhile, finite-difference method has higher accuracy than finite-element method in fixed-open case.}
  \label{fig:constant-v-subsonic}
\end{figure}

\section{Constant Velocity Case - Supersonic}
\begin{figure}[H]
  \begin{subfigure}{0.45\textwidth}
    \centering
    \includegraphics[width=0.9\linewidth]{../../thesis/img/numerical-experiments/fixed-fixed/constant-v-v0=1.5}
    \caption{Dirichlet boundary, filtered modes are stable.}
  \end{subfigure}%
  \begin{subfigure}{0.45\textwidth}
    \includegraphics[width=\linewidth]{../../thesis/img/numerical-experiments/fixed-open/constant-v-v0=1.5}
    \caption{Fixed-open boundary, all modes are unstable.}
  \end{subfigure}
  \caption{Showing the first 5 eigenvalues. In the Dirichlet boundary case, all methods are close to the exact eigenvalues. Meanwhile, finite-difference method has higher accuracy than finite-element method in fixed-open case.}
  \label{fig:constant-v-supersonic}
\end{figure}

\section{Error}
Because the existence of exact solution to problems Eq.(\ref{eq:polynomial-eigenvalue-problem}). The case with constant velocity profile is used as a sanity check. It allows us to verify the correctness of each method's implementation. This also serves as a reference to the accuracy spectral methods can achieve.

From Fig.\ref{fig:constant-v-subsonic} and Fig.\ref{fig:constant-v-supersonic}, we see that the order of growth rates is about $~10^{-14}$ for both subsonic and supersonic cases if the boundary condition is Dirichlet. We will use it a reference to the accuracy of our numerical methods. If a method produces growth rates with order close to $10^{-14}$, we consider the growth rates to be 0.

\begin{table} [H]
	\centering
	\caption{Relative error of each eigenvalue.}
	\begin{tabular}{|c|c|c|c|c|c|}
		\hline
		$v_0=0.5$   & 1 & 2 & 3 & 4 & 5 \\
		\hline
		FD & 2.827e-05 & 1.130e-04 & 2.541e-04 & 4.512e-04 & 7.040e-04 \\
		\hline
		FE & 0.005 & 0.005 & 0.006 & 0.008 & 0.010  \\
		\hline
		SE & 2.896e-05 & 1.157e-04 & 2.603e-04 & 4.626e-04 & 7.217e-04 \\
		\hline
	\end{tabular}
	\begin{tabular}{|c|c|c|c|c|c|}
		\hline
		$v_0=1.5$   & 1 & 2 & 3 & 4 & 5 \\
		\hline
		FD & 0.001 & 0.005 & 0.010 & 0.019 & 0.030 \\
		\hline
		FE & 0.006 & 0.010 & 0.019 & 0.029 & 0.043  \\
		\hline
		SE & 0.001 & 0.005 & 0.011 & 0.019 & 0.030 \\
		\hline
	\end{tabular}
	\label{table:eigenvalue-error-fixed-fixed}
\end{table}

\begin{table} [H]
	\centering
	\caption{Relative error of each eigenvalue. Notice that the ground mode for subsonic case is non-zero.}
	\begin{tabular}{|c|c|c|c|c|c|}
		\hline
		$v_0=0.5$   & 0 & 1 & 2 & 3 & 4 \\
		\hline
		FD & 1.209e-05 & 3.458e-05 & 5.775e-05 & 8.153e-05 & 1.061e-04 \\
		\hline
		FE & 8.090e-05 & 2.007e-04 & 2.981e-04 & 6.596e-04 & 1.821e-03  \\
		\hline
	\end{tabular}
	\begin{tabular}{|c|c|c|c|c|c|}
		\hline
		$v_0=1.5$   & 1 & 2 & 3 & 4 & 5 \\
		\hline
		FD & 9.163e-05 & 2.435e-04 & 4.833e-04 & 8.160e-04 & 1.243e-03 \\
		\hline
		FE & 4.431e-04 & 7.924e-04 & 1.516e-03 & 3.103e-03 & 8.001e-03  \\
		\hline
	\end{tabular}
	\label{table:eigenvalue-error-fixed-open}
\end{table}

\begin{figure}[H]
	\centering
	\begin{subfigure}{0.5\textwidth}
		\includegraphics[width=\linewidth]{../../thesis/img/numerical-experiments/fixed-open/constant-v-v0=0.5}
		\caption{All modes are stable.}
	\end{subfigure}%
	\begin{subfigure}{0.5\textwidth}
		\includegraphics[width=\linewidth]{../../thesis/img/numerical-experiments/fixed-open/constant-v-v0=1.5}
		\caption{All modes are unstable.}
	\end{subfigure}
	\caption{Showing the first 5 eigenvalues of each method. Finite-difference method has much better accuracy than finite-element method.}
	\label{fig:constant-v-fixed-open}
\end{figure}


\section{Subsonic Case}
\begin{figure} [H]
  \centering
  \begin{subfigure}{0.45\textwidth}
    \centering
    \includegraphics[width=\linewidth]{../../thesis/img/numerical-experiments/fixed-fixed/subsonic-v}
    \caption{Dirichlet boundary, all modes are stable.}
  \end{subfigure}%
  \begin{subfigure}{0.45\textwidth}
    \includegraphics[width=\linewidth]{../../thesis/img/numerical-experiments/fixed-open/subsonic-v}
    \caption{The ground mode is unstable, other modes are stable.}
  \end{subfigure}
  \caption{Showing the first 5 modes. It suggests that the subsonic flow in magnetic nozzle is stable.}
\end{figure}

\section{Supersonic Case}
\begin{figure} [H]
  \centering
  \begin{subfigure}{0.45\textwidth}
    \centering
    \includegraphics[width=\linewidth]{../../thesis/img/numerical-experiments/fixed-fixed/supersonic-v}
    \caption{Dirichlet boundary, filtered modes are stable.}
  \end{subfigure}%
  \begin{subfigure}{0.45\textwidth}
    \centering
    \includegraphics[width=\linewidth]{../../thesis/img/numerical-experiments/fixed-open/supersonic-v}
    \caption{Fixed-open boundary, all modes are unstable.}
  \end{subfigure}
  \caption{This suggests that the supersonic flow is stable if the boundary is Dirichlet and unstable if the boundary is left-fixed-right-open.}
\end{figure}


\section{Accelerating Case}
Starting from the singular point, we shoot the solution to the left boundary. We find the set of eigenvalues such that $\tilde{v}(-1)=0$. With these eigenvalues, we can extend the solution to the supersonic region $(0,1]$. The first five eigenvalues are drawn in the graph.
\begin{figure} [H]
	\centering
	\includegraphics[width=0.7\linewidth]{../../thesis/img/numerical-experiments/accelerating-v}
	\caption{All modes are stable.}
	\label{fig:accelerating-v}
\end{figure}

\chapter{Discussion} \label{chap:discussion}
\section{Implications of the Results}
The results tell us the subsonic and accelerating plasma flow is stable.

\section{Limitations of the methods}
\subsection{Spectral Method}
The spectral method suffers the spectral pollution. For now there is no automatic ways to filter spurious modes other than doing convergence test and pick up the convergent eigenvalues manually by ourselves. We believe there is a discretization scheme that is spectral pollution free. In fact, we made optimistic guess based on the fact that normal form of Eq.(\ref{eq:polynomial-eigenvalue-problem}) with constant velocity profile is pollution free.

\subsection{Shooting Method}
The shooting method is not exhaustive due to the nature of root finding algorithm. A root can only be found if the initial guess of the root is close enough to the actual root. To work around this issue, we have to perform a grid search on the complex plane. This way we can only survey the low frequency region due to the finiteness of computing resources. The conclusion for cases with transonic velocity profiles is true only for low frequency region.

\section{Conclusion}
In chapter 3, we derived the linearized equations of motion of the flow in one dimensional magnetic nozzle. Furthermore, we rewrite the linearized governing equations as an eigenvalue problem. Using the spectral methods introduced in chapter 2, we discretized the operators of the problem. Hence, transforming it into an algebraic eigenvalue problem.

With the aid of computer, we are able to solve the algebraic eigenvalue problem. The results show that the flow in magnetic nozzle with Dirichlet boundary condition is stable except the case with decelerating velocity profile.

\bibliographystyle{plain}
\bibliography{references}
\appendix
\include{lambert-w-function}
\include{verification-of-analytical-solutions}

\end{document}
