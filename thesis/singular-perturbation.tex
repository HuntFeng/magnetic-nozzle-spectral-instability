\chapter{Singular Perturbation}
\section{What is a singular perturbation problem}

\section{Spectral Method Fails with Transonic Velocity Profiles}

\section{Connection of Singularity to Black Hole}
\subsection{Acoustic Analog to Tortoise Coordinate}

\section{Linearization at the Singularity}

\section{Shooting Method}
Shooting method can be used to solve eigenvalue problem with specified boundary values,
\begin{equation} \label{eq:boundary-eigenvalue-problem}
    g(\tilde{v}(z);\omega) = 0,
    \quad
    z_l \leq z \leq z_r,
    \quad
    \tilde{v}(z_l) = \tilde{v}_l, \tilde{v}(z_r) = \tilde{v}_r
\end{equation}
where $\omega$ is the eigenvalue to be solved.

Suppose an eigenvalue problem can be formulated as
\[ \dv{z}\mathbf{u} = \mathbf{f}(\mathbf{u},z;\omega),
    \quad
    z_l<z<z_r,
    \quad
    \mathbf{u}(z_l) = \mathbf{u}_l
\]
where $\mathbf{u}\in\mathbb{R}^2$. Fixed an $\omega$, we can approximate $\mathbf{u}(z_r)$ by applying algorithms such as Runge-Kutta or Leap-frog.

Define $F$ by $F(\mathbf{u}_l;\omega)=\tilde{v}(z_r;\omega)$. This function $F$ takes in the initial value $\mathbf{u}_l$ and a fixed $\omega$, and outputs the "landing point" $\tilde{v}(z_r;\omega)$. If $\omega$ is an eigenvalue of Eq.(\ref{eq:boundary-eigenvalue-problem}), then $\tilde{v}(z_r;\omega) = \tilde{v}_r$. Now we can find eigenvalues to Eq.(\ref{eq:boundary-eigenvalue-problem}) by solving the roots to the scalar equation
\[h(\omega) = F(\mathbf{u}_l;\omega) - \tilde{v}_r\]

Having this higher view of shooting method in mind, we first transform Eq.(\ref{eq:polynomial-eigenvalue-problem}) to a IVP,
\begin{align*}
    v' & = u                        \\
    u' & = \frac{-1}{1-v_0^2}\left[
        \omega^2v + 2i\omega(v_0+v_0'v) - \left(3v_0 - \frac{1}{v_0}\right)v_0'u - \left(1-\frac{1}{v_0}^2\right)(v_0')^2v - \left(v_0+\frac{1}{v_0}v_0'' v\right)
        \right]
\end{align*}
and $v(0)=c_0=1,u(0)=c_1=(2i\omega v_0'-2v_0'')/2v_0'$. Moreover, $u'(0)=c_2=-((v_0')^4+(2i\omega v_0' + (v_0')^2 - v_0'')(i\omega v_0' - v_0''))/(v_0'(2i\omega-6v_0'))$.

In order to get initially value for cases with transonic velocity profiles, we need to expand the solution at the singularity.

\subsection{Expansion at Singularity}
If the equilibrium velocity profile $v_0$ is a transonic profile, then $v_0(0) = 1$ at the throat of the magnetic mirror configuration. This is a singularity. More specifically, it is a regular singular point.

In order to supply initial values to shooting method, we need to expand Eq.(\ref{eq:polynomial-eigenvalue-problem}) at the singularity.

Since
\begin{equation}
    \begin{aligned}
        1-v_0^2              & = -2v_0'(0)z    \\
        3v_0 + \frac{1}{v_0} & = 4 + 2v_0'(0)z \\
        1-\frac{1}{v_0^2}    & = 2v_0'(0)z     \\
        v_0 + \frac{1}{v_0}  & = 2
    \end{aligned}
\end{equation}

Then Eq.(\ref{eq:polynomial-eigenvalue-problem}) becomes
\begin{equation}
    \begin{aligned}
        - & 2v_0'(0)z\pdv[2]{\tilde{v}}{z}                                              \\
        + & [2i\omega - 4v_0'(0) + (2i\omega - 2v_0'(0))z]\pdv{\tilde{v}}{z}            \\
        + & \left[\omega^2 + 2i\omega v_0'(0) - 2v_0''(0) - 2v_0'(0)^3z\right]\tilde{v}
        = 0
    \end{aligned}
    \label{eq:perturbed-equation-full}
\end{equation}

Since we are expanding at $z=0$, we drop all $z$ terms except the first term (second-order derivative term).

\[ - 2v_0'(0)z\pdv[2]{\tilde{v}}{z}
    + (2i\omega - 4v_0'(0))\pdv{\tilde{v}}{z}
    + (\omega^2 + 2i\omega v_0'(0) - 2v_0''(0))\tilde{v}
    = 0 \]

Dividing by the first coefficient, we have
\begin{equation}
    z\pdv[2]{\tilde{v}}{z} + a\pdv{\tilde{v}}{z} + b\tilde{v} = 0
    \label{eq:perturbed-equation}
\end{equation}
where
\[ a = \frac{2i\omega - 4v_0'(0)}{-2v_0'(0)}; \quad
    b = \frac{\omega^2 + 2i\omega v_0'(0) - 2v_0''(0)}{-2v_0'(0)}
\]

Use Frobenius method, we assume $\tilde{v} = \sum_{n\geq 0}c_nz^{n+r}$, plug Eq.(\ref{eq:perturbed-equation}) we have
\[ \sum_{n \geq 0} (n+r)(n+r+1) c_n z^{n+r-1} + a(n+r)c_nz^{n+r-1} + bc_nz^{n+r} = 0\]
Shift the power of the last term we get
\[ \sum_{n \geq 0} (n+r)(n+r+1) c_n z^{n+r-1} + a(n+r)c_nz^{n+r-1} + \sum_{n \geq 1} bc_{n-1}z^{n+r-1} = 0\]

Setting $n=0$, we get the indicial equation
\[ c_0 r(r-1) + c_0 ar = 0 \Rightarrow c_0r(r+a-1) = 0 \]
We get two different roots, $r=0$ and $r=1-a$. They correspond to finite solution and diverging solution near the singularity, respectively.

The coefficients are given by recurrence relation
\[ (n+r)(n+r-1)c_n + a(n+r)c_n + bc_{n-1} = 0
    \Rightarrow
    c_n = \frac{-bc_{n-1}}{(n+r)(n+r-1+a)}
\]
Solving this relation we get explicit expression for $c_n$,
\begin{equation}
    \begin{aligned}
        c_n & = \frac{(-1)^n b^n c_0}{\prod_{k=0}^{n-1} (n+r-k)(n+r-1+a-k)}              \\
            & = (-1)^n b^n c_0 \frac{\Gamma(r+1)\Gamma(r+a)}{\Gamma(n+r+1)\Gamma(n+r+a)}
    \end{aligned}
    \label{eq:coefficient}
\end{equation}

\begin{itemize}
    \item Worth to mention that the diverging solution goes like
          \[ \tilde{v}(z) \sim z^{1-a} = z^{-1-\omega_i/v_0'(0)}z^{i\omega_r/v_0'(0)}  \]
          where $\omega = \omega_r + i\omega_i$. Meaning that the solution will be divergent iff $\omega_i > -v'(0)$.
    \item Dropping the $z$ terms in Eq.(\ref{eq:perturbed-equation-full}) has no effect on the first order correction ($\tilde{v}$ is the same up to $z$ term).
\end{itemize}
