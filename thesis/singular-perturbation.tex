\chapter{Singular Perturbation} \label{chap:singular-perturbation}
This chapter is dedicated to analyze the polynomial eigenvalue problem with transonic velocity profiles. We will first show the existence of the singularity of Eq.(\ref{eq:polynomial-eigenvalue-problem}), then we will discuss the concept of singular perturbation and the way we solve the problem.

\section{Presence of Singularity in Transonic cases} \label{sec:presence-of-singularity}
In order to see the existence of the singularity, we rearrange the terms the polynomial eigenvalue problem, Eq.(\ref{eq:polynomial-eigenvalue-problem}),
\begin{equation} \label{eq:singular-perturbation-problem}
	\begin{aligned}
		  & (1-v_0^2)\pdv[2]{\tilde{v}}{z}                                                                                                                                        \\
		+ & \left[2i\omega v_0 - \left(3v_0 + \frac{1}{v_0}\right)\pdv{v_0}{z}\right]\pdv{\tilde{v}}{z}                                                                           \\
		+ & \left[\omega^2 + 2i\omega\pdv{v_0}{z} -\left(1 - \frac{1}{v_0^2}\right)\left(\pdv{v_0}{z}\right)^2 - \left(v_0 + \frac{1}{v_0}\right)\pdv[2]{v_0}{z} \right]\tilde{v} \\
		= & 0
	\end{aligned}
\end{equation}
This is a second order ordinary differential equation defined on region $[-1,1]$.

For transonic (accelerating and decelerating) velocity profiles (Fig.\ref{fig:velocity-profiles}), the plasma flow is at sonic point at the throat of the nozzle, $v_0(0)=1$. Therefore, the highest order term vanishes at $z=0$. This causes the failure of spectral method, see Fig.(\ref{chap:discussion}). It can be proved that $z=0$ is a regular singular point.

\begin{figure} [htbp]
	\centering
	\includegraphics[width=0.7\textwidth]{img/results-bad-accelerating-v}
	\caption{An attempt to solve the polynomial eigenvalue problem, Eq.(\ref{eq:polynomial-eigenvalue-problem}) using finite-difference. Eigenfunctions are squeezed to the center of the nozzle due to the existence of the singularity at $z=0$.}
	\label{fig:failure-of-spectral-method}
\end{figure}

\section{Connection of Singularity to Black Hole}
\subsection{Acoustic Analog to Tortoise Coordinate}

\section{Singular Perturbation Problem}
Consider Eq.(\ref{eq:polynomial-eigenvalue-problem}) as a boundary value problem by fixing the value of $\omega$,
The boundary value problem Eq.(\ref{eq:singular-perturbation-problem}) is defined on region $[-1,1]$, but the boundary values are defined on $z=-1$ and $z=0$. This is because to extract a regular solution near the singularity, we need to assume the solution is finite at the throat of the nozzle, $z=0$. This condition serves as a boundary value to the problem. Together with the boundary condition at the entrance of the nozzle, $\tilde{v}(-1) = 0$, the solutions to the boundary value problem, Eq.(\ref{eq:singular-perturbation-problem}) is fully determined up to a set of eigenvalues.

As we discuss in the earlier section, Sec.\ref{sec:presence-of-singularity}, $(1-v_0^2)$ is 0 at the nozzle throat $z=0$ since $v_0(z)$ is now a transonic velocity profile. This makes the problem a first order ordinary differential equation,
\[
	\left( 2i\omega - 4\eval{\pdv{v_0}{z}}_{z=0} \right)\pdv{\tilde{v}}{z}
	+ \left[2i\omega\eval{\pdv{v_0}{z}}_{z=0} - 2\eval{\pdv[2]{v_0}{z}}_{z=0} \right]\tilde{v}  = 0
\]
It is clear that the solution to this first order ODE has a totally different characteristics than the solution in the neighborhood of $z=0$. That means we cannot simply set $(1-v_0^2)$ to 0 and get an asymptotic approximation to the solution of Eq.(\ref{eq:singular-perturbation-problem}) near $z=0$. This is exactly the characteristics of singular perturbation problem.

\section{Expansion at Singularity}
The singularity is a regular singular point, we are able to extract finite solution near the singularity. In order to do so, we need to expand terms in Eq.(\ref{eq:singular-perturbation-problem}) about the singularity.

The first task is to linearize the terms with $v_0$ about the singularity. The linearization of $v_0(z) = 1 + v_0'(0)z$ is a good approximation to the original function $v_0(z)$ because the transonic velocity profiles are straight in the neighborhood of $z=0$, as we can see from Fig.\ref{fig:velocity-profiles}. Therefore, through some simple algebra we obtain,
\begin{equation}
	\begin{aligned}
		1-v_0^2              & = -2v_0'(0)z    \\
		3v_0 + \frac{1}{v_0} & = 4 + 2v_0'(0)z \\
		1-\frac{1}{v_0^2}    & = 2v_0'(0)z     \\
		v_0 + \frac{1}{v_0}  & = 2
	\end{aligned}
\end{equation}

Then Eq.(\ref{eq:singular-perturbation-problem}) becomes
\begin{equation} \label{eq:perturbed-equation-full}
	\begin{aligned}
		- & 2v_0'(0)z\pdv[2]{\tilde{v}}{z}                                              \\
		+ & [2i\omega - 4v_0'(0) + (2i\omega - 2v_0'(0))z]\pdv{\tilde{v}}{z}            \\
		+ & \left[\omega^2 + 2i\omega v_0'(0) - 2v_0''(0) - 2v_0'(0)^3z\right]\tilde{v}
		= 0
	\end{aligned}
\end{equation}

In fact, we can further simply the equation by dropping all $z$ terms except the first term (second-order derivative term). It can be shown that dropping the $z$ terms in Eq.(\ref{eq:perturbed-equation-full}) does not affect the first order correction ($\tilde{v}$ is the same up to $z$ term), it is an acceptable approximation.

\[ - 2v_0'(0)z\pdv[2]{\tilde{v}}{z}
	+ (2i\omega - 4v_0'(0))\pdv{\tilde{v}}{z}
	+ (\omega^2 + 2i\omega v_0'(0) - 2v_0''(0))\tilde{v}
	= 0 \]

Dividing by the first coefficient, we have
\begin{equation}
	z\pdv[2]{\tilde{v}}{z} + a\pdv{\tilde{v}}{z} + b\tilde{v} = 0
	\label{eq:perturbed-equation}
\end{equation}
where
\[ a = \frac{2i\omega - 4v_0'(0)}{-2v_0'(0)}; \quad
	b = \frac{\omega^2 + 2i\omega v_0'(0) - 2v_0''(0)}{-2v_0'(0)}
\]

Use Frobenius method, we assume the velocity perturbation can be written as a power series in $z$,  $\tilde{v} = \sum_{n\geq 0}c_nz^{n+r}$. By substituting the power series into Eq.(\ref{eq:perturbed-equation}) we have
\[ \sum_{n \geq 0} (n+r)(n+r+1) c_n z^{n+r-1} + a(n+r)c_nz^{n+r-1} + bc_nz^{n+r} = 0\]
Shift the power of the last term we get
\[ \sum_{n \geq 0} (n+r)(n+r+1) c_n z^{n+r-1} + a(n+r)c_nz^{n+r-1} + \sum_{n \geq 1} bc_{n-1}z^{n+r-1} = 0\]

Setting $n=0$, we get the indicial equation
\[ c_0 r(r-1) + c_0 ar = 0 \Rightarrow c_0r(r+a-1) = 0 \]
We get two different roots, $r=0$ and $r=1-a$. They correspond to finite solution and diverging solution near the singularity, respectively.

The coefficients are given by recurrence relation
\[ (n+r)(n+r-1)c_n + a(n+r)c_n + bc_{n-1} = 0
	\Rightarrow
	c_n = \frac{-bc_{n-1}}{(n+r)(n+r-1+a)}
\]
Solving this relation we get explicit expression for $c_n$, $n\in\mathbb{N}$,
\begin{equation}
	\begin{aligned}
		c_n & = \frac{(-1)^n b^n c_0}{\prod_{k=0}^{n-1} (n+r-k)(n+r-1+a-k)}              \\
		    & = (-1)^n b^n c_0 \frac{\Gamma(r+1)\Gamma(r+a)}{\Gamma(n+r+1)\Gamma(n+r+a)}
	\end{aligned}
	\label{eq:coefficient}
\end{equation}

Therefore, we successfully extracted the regular solution (corresponding to the root $r=0$) in the form of power series,
\begin{equation} \label{eq:regular-solution}
	\begin{aligned}
		\tilde{v} & = c_0 + c_1z + c_2z^2 + \cdots                               \\
		          & = c_0 - c_0\frac{b}{a}z + c_0\frac{b^2}{2a(1+a)}z^2 + \cdots
	\end{aligned}
\end{equation}

It is worth to mention that the diverging solution (corresponding to the root $r=a$) goes like
\[ \tilde{v}(z) \sim z^{1-a} = z^{-1-\omega_i/v_0'(0)}z^{i\omega_r/v_0'(0)}  \]
where $\omega = \omega_r + i\omega_i$. Meaning that the divergent solution will start to diverge when $\omega_i > -v'(0)$.

\section{Shooting Method}
Shooting method can be used to solve eigenvalue problem with specified boundary values,
\begin{equation} \label{eq:boundary-eigenvalue-problem}
	g(\tilde{v}(z);\omega) = 0,
	\quad
	z_l \leq z \leq z_r,
	\quad
	\tilde{v}(z_l) = \tilde{v}_l, \tilde{v}(z_r) = \tilde{v}_r
\end{equation}
where $\omega$ is the eigenvalue to be solved.

Suppose an eigenvalue problem can be formulated as
\[ \dv{z}\mathbf{u} = \mathbf{f}(\mathbf{u},z;\omega),
	\quad
	z_l<z<z_r,
	\quad
	\mathbf{u}(z_l) = \mathbf{u}_l
\]
where $\mathbf{u}\in\mathbb{R}^2$. Fixed an $\omega$, we can approximate $\mathbf{u}(z_r)$ by applying algorithms such as Runge-Kutta or Leap-frog.

Define $F$ by $F(\mathbf{u}_l;\omega)=\tilde{v}(z_r;\omega)$. This function $F$ takes in the initial value $\mathbf{u}_l$ and a fixed $\omega$, and outputs the "landing point" $\tilde{v}(z_r;\omega)$. If $\omega$ is an eigenvalue of Eq.(\ref{eq:boundary-eigenvalue-problem}), then $\tilde{v}(z_r;\omega) = \tilde{v}_r$. Now we can find eigenvalues to Eq.(\ref{eq:boundary-eigenvalue-problem}) by solving the roots to the scalar equation
\[h(\omega) = F(\mathbf{u}_l;\omega) - \tilde{v}_r\]

Having this higher view of shooting method in mind, we first transform Eq.(\ref{eq:polynomial-eigenvalue-problem}) to a IVP,
\begin{align*}
	v' & = u                        \\
	u' & = \frac{-1}{1-v_0^2}\left[
		\omega^2v + 2i\omega(v_0+v_0'v) - \left(3v_0 - \frac{1}{v_0}\right)v_0'u - \left(1-\frac{1}{v_0}^2\right)(v_0')^2v - \left(v_0+\frac{1}{v_0}v_0'' v\right)
		\right]
\end{align*}
and $v(0)=c_0=1,u(0)=c_1=(2i\omega v_0'-2v_0'')/2v_0'$. Moreover, $u'(0)=c_2=-((v_0')^4+(2i\omega v_0' + (v_0')^2 - v_0'')(i\omega v_0' - v_0''))/(v_0'(2i\omega-6v_0'))$.

In order to get initially value for cases with transonic velocity profiles, we need to expand the solution at the singularity.

