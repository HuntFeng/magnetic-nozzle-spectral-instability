\chapter{Spectral Method} \label{chap:spectral-method}
\section{Spectral Theory in Finite Dimensional Normed Spaces}
Let $X$ be a finite dimensional normed space and $\hat{T}: X \to X$ a linear operator. Since any linear operator can be represented by a matrix, the spectral theory of $\hat{T}$ is essentially matrix eigenvalue theory. \cite{kreyszig_introductory_1978} Let $A$ be a matrix representation of $\hat{T}$, then we have the definition.

\begin{definition}
	An eigenvalue of a square matrix $A$ is a complex number $\lambda$ such that
	\[ Ax = \lambda x \]
	has a solution $x\neq 0$.This $x$ is called an \textbf{eigenvector} of $A$ corresponding to that eigenvalue $\lambda$.The set $\sigma(A)$ of all eigenvalues of $A$ is called the \textbf{spectrum} of $A$. Its complement $\rho(A) = \mathbb{C}-\sigma(A)$ in the complex plane is called the \textbf{resolvent} set of $A$.
\end{definition}

By choosing different bases in $X$, we can have different matrix representation of $\hat{T}$. We need to make sure the eigenvalues of a linear operator is independent of the basis chosen. Fortunately, a theorem ensures that.

\begin{theorem}
	All matrices representing a given linear operator $\hat{T}: X \to X$ on a finite dimensional normed space $X$ relative to various bases for $X$ have the same eigenvalues.
\end{theorem}


Moreover, we don't need to worry about the existence of eigenvalues of a linear operator. The following theorem shows the existence of them.
\begin{theorem}
	A linear operator on a finite dimensional complex normed space $X\neq{O}$ has at least one eigenvalue.
\end{theorem}


\section{Spectral Theory in Normed Spaces of Any Dimension}
Let $X\neq {0}$ be a complex normed space (could be any dimension), and $\hat{T}: D(\hat{T}) \to X$ with domain $D(\hat{T}) \subset X$. Again, we could define eigenvalues, and other related concepts in terms of the equation
\[ \hat{T}x = \lambda x \]

\begin{definition}
	Let $\hat{T}\neq{0}$ be a complex normed space and $\hat{T}: D(\hat{T}) \to X$ a linear operator with domain $D(\hat{T})\subset X$. A \textbf{regular value} $\lambda$ of $\hat{T}$ is a complex number such that
	\begin{itemize}
		\item [(R1)] $(\hat{T}-\lambda I)^{-1}$ exists,
		\item [(R2)] $(\hat{T}-\lambda I)^{-1}$ is bounded,
		\item [(R3)] $(\hat{T}-\lambda I)^{-1}$ is defined on a set which is dense in $X$,
	\end{itemize}
	
	The \textbf{resolvent set} $\rho(\hat{T})$ of $\hat{T}$ is the set of all regular values $\lambda$ of $\hat{T}$. Its complement $\sigma(\hat{T}) = \mathbb{C} - \rho(\hat{T})$ in the complex plane $\mathbb{C}$ is called the \textbf{spectrum} of $\hat{T}$, and a $\lambda\in \sigma(\hat{T})$ is called a \textbf{spectral value} of $\hat{T}$. Furthermore, the spectrum $\sigma(\hat{T})$ is partitioned into three disjoint sets as follows.
	\begin{itemize}
		\item The \textbf{point spectrum} or \textbf{discrete spectrum} $\sigma_p(\hat{T})$ is the set such that $(\hat{T}-\lambda I)^{-1}$ does not exist. A $\lambda\in\sigma_p(\hat{T})$ is called an \textbf{eigenvalue} of $\hat{T}$.
		\item The \textbf{continuous spectrum} $\sigma_c(\hat{T})$ is the set such that $(\hat{T}-\lambda I)^{-1}$ exists and satisfies (R3) but not (R2), that is, $(\hat{T}-\lambda I)^{-1}$ is unbounded.
		\item The \textbf{residual spectrum} $\sigma_r(\hat{T})$ is the set such that $(\hat{T}-\lambda I)^{-1}$ exists (and may be bounded or not) but does not satisfy (R3), that is, the domain of $(\hat{T}-\lambda I)^{-1}$ is not dense in X.
	\end{itemize}
\end{definition}

In practice, the eigenvalue problem in infinite dimension is difficult. 
Therefore, the usual approach to the eigenvalue problem $\hat{T}x=\lambda x$ is to first discretize the operator $\hat{T}$ to an approximated matrix operator $T$, then the eigenvalue problem becomes,
\[ Tx = \lambda x \]

There are different ways to discretize the operator. For example, we can use finite difference, finite element and spectral element methods. 

One important thing we need to keep in mind is that, the discretized version of the eigenvalue problem can have eigenvalues that are not in $\sigma(\hat{T})$. Those eigenvalues are called spurious eigenvalues, and this phenomenon is called spectral pollution. It is due to the improper discretization of the operators. We will discuss spectral pollution in the next section.

\section{Spectral Method}
Spectral method is one of the best tool to solve PDE and ODE problems. \cite{trefethen_spectral_2000}. The central idea of spectral method is by discretizing the equation, we can transform that to a linear system or an eigenvalue problem.
\subsection{Finite Difference}
Consider equally spaced nodes on domain $[-1,1]$, $\{x_1, x_2, \dots, x_N\}$ with $x_{j+1}-x_{j} = h$ for each $j$, and the set of corresponding function values, $\{ f_1, f_2, \dots, f_N \}$. We can approximate the derivatives using second-order central difference formulas
\[ 
\pdv{f}{z} = \frac{f_{j+1} - f_{j-1}}{2h}
\qquad
\pdv[2]{f}{z} = \frac{f_{j+1} -2f_{j} +f_{j-1}}{h^2}
\]

We can discretize the differentiation operators to the following matrices
\[ 
\pdv{z} \rightarrow D = \frac{1}{2h}\begin{bmatrix}
	0 & 1 & 0 & \dots & 0 \\
	-1 & \ddots & \ddots & \ddots & \vdots \\ 
	0 & \ddots & \ddots & \ddots & 0 \\
	\vdots & \ddots & \ddots & \ddots & 1 \\
	0 & \dots & 0 & -1 & 0 
\end{bmatrix} 
\qquad
\pdv[2]{z} \rightarrow  D^2 = \frac{1}{h^2}\begin{bmatrix}
	-2 & 1 & 0 & \dots & 0 \\
	1 & \ddots & \ddots & \ddots & \vdots \\ 
	0 & \ddots & \ddots & \ddots & 0 \\
	\vdots & \ddots & \ddots & \ddots & 1 \\
	0 & \dots & 0 & 1 & -2 
\end{bmatrix} 
\]

\subsection{Spectral Element}
Suppose the basis functions are $\{u_k(z)\}_{k=1}^\infty$, then the eigenfunction $\tilde{v}$ can be approximated by finite amount of them, $\tilde{v}(z) = \sum_{k=1}^N c_ku_k(z)$ where $c_k$ are coefficients to be determined.

Then by multiplying $u_{i}$ to any term and integrate through the domain, we can discretize the equation. Using the notation of inner product $(f,g)=\int_{-1}^{1} dz fg$, we see that

\[ \int_{-1}^{1} dz \; u_i\tilde{v} = \sum_{j}(u_i,u_j)c_j \]
\[ \int_{-1}^{1} dz \; u_i\pdv{\tilde{v}}{z} = \sum_{j}\left(u_i,\pdv{u_j}{z}\right)c_j \]
\[ \int_{-1}^{1} dz \; u_i\pdv[2]{\tilde{v}}{z} = \sum_{j}\left(u_i,\pdv[2]{u_j}{z}\right)c_j \]

\subsection{Finite Element}
Finite-element method is a generalization of the spectral element method. We are allow to use a set of basis functions similar to spectral method in a cell. The region consists of many of these cells.

The formulation is the same as Eq.(\ref{eq:eigenvalue-problem-SE}). The only difference is that in finite-element we need to solve Eq.(\ref{eq:eigenvalue-problem-SE}) simultaneously for all cells.
